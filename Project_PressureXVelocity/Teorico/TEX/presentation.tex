\documentclass[xcolor=dvipsnames,10pt,aspectratio=169]{beamer}
%\documentclass[xcolor=dvipsnames,10pt]{beamer}
\usepackage{etex}
\usepackage{pgf,pgfarrows,pgfnodes,pgfautomata,pgfheaps,pgfshade}
\usepackage[absolute,overlay]{textpos}
%\usepackage{algorithm}
\usepackage{amsmath,amssymb}
\usepackage[utf8]{inputenc}
\usepackage{colortbl}
\usepackage{graphicx}
\usepackage[brazil]{babel}
\usepackage{tabularx}
\usepackage{multirow}
\usepackage{booktabs}
\usepackage{listings}
%\usepackage{multimedia}
\usepackage{animate}
\usepackage{xcolor}
\usepackage{array}
\usepackage{longtable}
\usepackage{makecell}
\usepackage{caption}
\usetheme{Madrid}
\usepackage{amsmath}
\usepackage{movie15}
\usepackage{tikz}
\usetikzlibrary{shapes.geometric,arrows,shadows}

%Definição dos layers.
\pgfdeclarelayer{background}
\pgfdeclarelayer{foreground}
\pgfsetlayers{background,main,foreground}



\lstset{ %
%	backgroundcolor=\color{white},   % choose the background color; you must add \usepackage{color} or \usepackage{xcolor}
%	basicstyle=\footnotesize,        % the size of the fonts that are used for the code
	basicstyle=\scriptsize,        % the size of the fonts that are used for the code
	breakatwhitespace=false,         % sets if automatic breaks should only happen at whitespace
	breaklines=true,                 % sets automatic line breaking
	captionpos=t,                    % sets the caption-position to bottom
	commentstyle=\color{mygreen},    % comment style
	deletekeywords={...},            % if you want to delete keywords from the given language
	escapeinside={\%*}{*)},          % if you want to add LaTeX within your code
	extendedchars=true,              % lets you use non-ASCII characters; for 8-bits encodings only, does not work with UTF-8
%	frame=single,                    % adds a frame around the code
	keepspaces=true,                 % keeps spaces in text, useful for keeping indentation of code (possibly needs columns=flexible)
	keywordstyle=\color{blue},       % keyword style
%	language=make,                 % the language of the code
	morekeywords={*,...},            % if you want to add more keywords to the set
%	numbers=left,                    % where to put the line-numbers; possible values are (none, left, right)
%	numbersep=5pt,                   % how far the line-numbers are from the code
	numberstyle=\tiny\color{mygray}, % the style that is used for the line-numbers
	rulecolor=\color{black},         % if not set, the frame-color may be changed on line-breaks within not-black text (e.g. comments (green here))
	showspaces=false,                % show spaces everywhere adding particular underscores; it overrides 'showstringspaces'
	showstringspaces=false,          % underline spaces within strings only
	showtabs=false,                  % show tabs within strings adding particular underscores
	stepnumber=2,                    % the step between two line-numbers. If it's 1, each line will be numbered
}

\definecolor{mygreen}{rgb}{0,0.6,0}
\definecolor{mygray}{rgb}{0.5,0.5,0.5}
\definecolor{mymauve}{rgb}{0.58,0,0.82}

\usecolortheme{beaver}
\newcommand{\ul}{\underline}
\setbeamertemplate{footline}{\scriptsize{\vspace*{0.3cm}\hspace*{15cm}\insertframenumber\,/\,\inserttotalframenumber}}
\setbeamertemplate{caption}[numbered]
\setbeamerfont{caption}{size=\fontsize{8}{5}}

\setbeamercolor{block title}{	bg=Sepia , fg = White}
\setbeamercolor{block body}{bg=Brown!15, fg=Sepia }
\setbeamercolor{item projected}{bg=Sepia, fg=White}
\setbeamercolor{number projected}{bg = Black}

%declara as imagens usadas no layout do slide
\pgfdeclareimage[height=0.8cm]{mflab}{figuras_presentation_template/logo_mflab_transparente.png}
\pgfdeclareimage[height=1.0cm]{logoufu}{figuras_presentation_template/logo_ufu.jpg}
\pgfdeclareimage[height=1.0cm]{petro}{figuras_presentation_template/petrobras_2.png}

%posiciona o logotipo do MFLab
\setlength{\TPHorizModule}{1mm}
\setlength{\TPVertModule}{1mm}
\newcommand{\placelogomflab}
{
	\begin{textblock}{13}(150.0,0.0)
		\pgfuseimage{mflab}
	\end{textblock}

% 	\begin{textblock}{13}(128.0,1.0)
% 		\pgfuseimage{logoufu}
% 	\end{textblock}

	\begin{textblock}{13}(150.0,70.0)
		\pgfuseimage{petro}
	\end{textblock}
}
%posiciona o logotipo do MFLab
\setlength{\TPHorizModule}{1mm}
\setlength{\TPVertModule}{1mm}
\newcommand{\placelogo}
{
	\begin{textblock}{13}(150.0,0.0)
		\pgfuseimage{mflab}
	\end{textblock}

% 	\begin{textblock}{13}(128.0,1.0)
% 		\pgfuseimage{logoufu}
% 	\end{textblock}

	\begin{textblock}{13}(0.0,80.0)
		\pgfuseimage{petro}
	\end{textblock}
}

% \setlength{\TPHorizModule}{1mm}
% \setlength{\TPVertModule}{1mm}
% \newcommand{\placelogomflab_titulo}
% {
% 	\begin{textblock}{13}(150.0,0.0)
% 		\pgfuseimage{mflab}
% 	\end{textblock}
%
% 	\begin{textblock}{13}(0.0,0.0)
% 		\pgfuseimage{lmest}
% 	\end{textblock}
%
% % 	\begin{textblock}{13}(128.0,1.0)
% % 		\pgfuseimage{logoufu}
% % 	\end{textblock}
%
% 	\begin{textblock}{13}(75.0,80.0)
% 		\pgfuseimage{petro}
% 	\end{textblock}
% }



%insere o logotipo da ufu em todos os slides
% \logo{\includegraphics[height=0.8cm]{figuras/layout_slide/petrobras.png}}

\title{Uma análise bidimensional de escoamentos incompressíveis.}

\author{ Felipe J. O. Ribeiro \\ \and \\ Orientador: Prof. Dr. Aristeu da Silveira Neto}

%\date{\tiny{02 de dezembro de 2015}}
\date{\tiny{\today}}
% \newcolumntype{M}[1]{>{\raggedright\arraybackslash}b{#1}}
% \newcolumntype{N}{@{}m{0pt}@{}}
% \newcolumntype{M}{>{\begin{minipage}[b]{3cm}\raggedright{}}c<{\end{minipage}\minrowheight}}
% \setlength\extrarowheight{5pt}
\newcolumntype{C}[1]{>{\centering\let\newline\\\arraybackslash\hspace{0pt}}m{#1}}


\begin{document}




	\begin{frame}\placelogomflab
		\frametitle
		{ \vfill
			\centering
			{
			\small{Universidade Federal de Uberlândia}\\
%			\small{Programa de Pós-Graduação em Engenharia Mecânica}\\
			\small{Laboratório de Mecânica dos Fluidos}\\
			}
		}
		\maketitle
	\end{frame}





\section<presentation>*{Sumário}

	\begin{frame}
		\frametitle{Sumário}\placelogomflab
		{\scriptsize \tableofcontents}
	\end{frame}





	\AtBeginSection[]
	{
	\begin{frame}<beamer>
		\frametitle{Sumário}\placelogomflab
		{\scriptsize \tableofcontents[current,currentsection]}
	\end{frame}
	}
	\AtBeginSubsection[]
	{
	\begin{frame}<beamer>
		  \frametitle{Sumário}\placelogomflab
		  {\scriptsize \tableofcontents[current,currentsubsection]}
	\end{frame}
	}





%%%%%%%%%%%%%%%%%%%%%%%%%%%%%%%%%%%%%%%%%%%%%%%%%%%%%%%%%%%%%%%%%%%%%%%%%%%%%%%%%%%%%%%%%%%%%%
\section{Introdução}

	\begin{frame}
		\frametitle{Objetivos}

		\centering
		O presente texto procura documentar o desenvolvimento deste trabalho de pesquisa com o máximo de detalhes nas metodologias possível. Dês dos desenvolvimentos matemáticos e teóricos aos andamentos nos aspectos técnicos e organizacionais.

		\vspace{0.5cm}

		\flushleft
		Os tópicos centrais deste estudo são:\\
		\quad $\bullet$ Desenvolvimento teórico e matemático do escoamento bidimensional em cavidade.\\
		\quad $\bullet$ Desenvolvimento das conversões dos parâmetros do Poiseuille para a cavidade.\\
		\quad $\bullet$ Aperfeiçoamento dos protocolos de visualização de dados com OpenGL.\\
		\quad $\bullet$ Desenvolvimento das rotinas em MPI no domínio bidimensional.\\

	\end{frame}





	\begin{frame}
		\frametitle{Estrutura de arquivos}
		\begin{tabular}{c c}
			{
				\begin{minipage}[h!]{0.15\textwidth}
					\centering
					\small
					Neste projeto, dês do início, será dada uma atenção especial à organização dos documentos, de forma a simplificar o andamento das atividades de documentação.
					\vspace{6cm}
				\end{minipage}
			}&{
				\includegraphics[trim={0.0cm 0.0cm 0.0cm 0.0cm},clip=true, scale = 0.435]{../../arquivos.pdf}
			}
		\end{tabular}

	\end{frame}





	\begin{frame}
		\frametitle{Documento "Registro.tex"}
		\begin{tabular}{c c}
			{
				\begin{minipage}[h!]{0.4\textwidth}
					\centering
					\small
					Por questões de adequações á formatação em textos científicos, além desta apresentação também serão registrados os andamentos nesse documento em inglês. Ele segue o template da ASME para textos científicos. Assim, pretende-se já desenvolver as rotinas de visualização dos resultados de acordo com estas convenções. De forma a facilitar a criação de documentos para submissão em artigos e tornando este processo mais ágil.
					\vspace{6cm}
				\end{minipage}
			}&{
				\includegraphics[page = 1,trim={0.0cm 12.0cm 0.0cm 0.0cm},clip=true, scale = 0.4]{registro.pdf}
			}
		\end{tabular}

	\end{frame}





%	\begin{frame}
%		\frametitle{Levante bibliográfico}
%		\begin{minipage}[h!]{0.33\textwidth}
%			\begin{figure}
%				\centering
%				\includegraphics[page = 1,trim={0.0cm 0.0cm 0.0cm 0.0cm},clip=true, scale = 0.15]{Referencias/An_introduction_to_Computational}
%				\caption{Possui as referencias aos métodos matemáticos e numéricos de modelagem e discretização da fluidodinâmica ao domínio bidimensional.}
%			\end{figure}
%			\vspace{6cm}
%		\end{minipage}
%		\begin{minipage}[h!]{0.33\textwidth}
%			\begin{figure}
%				\centering
%				\includegraphics[page = 1,trim={0.0cm 0.0cm 0.0cm 0.0cm},clip=true, scale = 0.165]{Referencias/MPI_Guide_FORTRAN.png}\\
%				\caption{Um guia para se aprender a desenvolver em FORTRAN as rotinas em paralelo, com passagem de mensagem entre os processos.}
%			\end{figure}
%			\vspace{6cm}
%		\end{minipage}
%		\begin{minipage}[h!]{0.325\textwidth}
%			\begin{figure}
%				\centering
%				\includegraphics[page = 1,trim={1.0cm 1.0cm 1.0cm 0.0cm},clip=true, scale = 0.165]{Referencias/Projeto_cebeci}\\
%				\caption{Artigo previamente publicado, para resgate dos ajustes paramétricos.}
%			\end{figure}
%			\vspace{6cm}
%		\end{minipage}
%
%	\end{frame}





	\begin{frame}
		\frametitle{Análise térmica bidimensional e tridimensional da cavidade}
		$\bullet$ A convecção natural é um fenômeno clássico de escoamento em cavidade. Ele representa muitas situações industriais e do dia a dia.
		Para se simular tal fenômeno um domínio bi ou tridimensional é necessário. Dessa forma torna-se uma extensão natural do trabalho desenvolvido até o momento.

		Apesar de lidar com mudanças de massa dos fluidos, Se considerará o escoamento como incompressível:

		- Desenvolver modelos 2D e 3D representativos.

		- Implementar métodos de otimização conforme necessários, como mpi e opengl.

		\begin{figure}[h!]
			\centering
			\includegraphics[trim = {1.7cm 2cm 0 1cm}, clip , angle=0, scale=0.50]{images/NaturalConvectionFromNet}
			\caption{Convecção natural.}
		\end{figure}

	\end{frame}





%%%%%%%%%%%%%%%%%%%%%%%%%%%%%%%%%%%%%%%%%%%%%%%%%%%%%%%%%%%%%%%%%%%%%%%%%%%%%%%%%%%%%%%%%%%%%%
\section{Modelo físico}

	\begin{frame}
		\frametitle{Hipóteses e considerações sobre o sistema}
		\begin{tabular}{c c}
			{
				\begin{minipage}{0.4\textwidth}
					\small
					\centering
					O problema escolhido fora o da cavidade. Ele consiste em uma quantidade de fluido confinada, como na Fig.\ref{sistema_termico_1}:

					\vspace{0.2cm}

					\flushleft
					As considerações feitas foram:\\
					\quad $\bullet$ O fluido é incompressível e newtoniano, sendo que sua massa específica varia somente em função da temperatura.\\
					\quad $\bullet$ O sistema foi considerado auto-similar na coordenada $z$. Sem velocidade ou fluxo térmico nesta direção. \\
					\quad $\bullet$ As superfícies perpendiculares ao eixo $y$ são isoladas termicamente.\\
					\quad $\bullet$ Considera-se uma fonte fria e uma fonte quente nas superfícies perpendiculares ao eixo $x$.\\
				\end{minipage}
			}&{
				\begin{minipage}{0.5\textwidth}
					\begin{figure}
						\centering
						\includegraphics[page = 1,trim={6.0cm 2.0cm 6.0cm 1.5cm},clip=true, scale = 0.4]{images/Cavidade_fisico_1.pdf}
						\caption{Sistema físico da cavidade.}
						\label{sistema_termico_1}
					\end{figure}
				\end{minipage}
			}
		\end{tabular}

	\end{frame}







%%%%%%%%%%%%%%%%%%%%%%%%%%%%%%%%%%%%%%%%%%%%%%%%%%%%%%%%%%%%%%%%%%%%%%%%%%%%%%%%%%%%%%%%%%%%%%
\section{Estudos introdutórios}




	\begin{frame}
		\frametitle{Domínio bidimensional com advecção imposta}
		\begin{minipage}[h!]{0.49\textwidth}
			$\bullet$ Converter código bidimensional antigo em MatLab para c++.\\
			$\bullet$ Desenvolver advecção, com velocidades em $x$ e $y$ impostas para todo o domínio.\\
			$\bullet$ Desenvolver biblioteca visual para este estudo com openGL.\\
			$\bullet$ Converter para o fortran os códigos.\\
			$\bullet$ Experimentar plataforma openGL com fortran.\\
			$\bullet$ Estudar acoplamento velocidade pressão.\\
		\end{minipage}
		\begin{minipage}[h!]{0.49\textwidth}
			\begin{figure}[h!]
				\centering
				\includegraphics[trim = {0cm 1cm 0 1cm}, clip , angle=0, scale=0.15]{images/ImagemProfessor1}
				\caption{Esquema do professor.}
			\end{figure}
		\end{minipage}
	\end{frame}



	\begin{frame}
		\frametitle{Modelo matemático diferencial}
		$\bullet$ A equação de balanço da energia térmica foi desenvolvida como base, com termos advectivos.\\
		$\bullet$ A dimensão $z$ foi ignorada, considerando-se auto similaridade neste eixo.\\

		\begin{equation}
			\frac{\partial T}{\partial t} + u \frac{\partial T}{\partial x} + v \frac{\partial T}{\partial y} = \alpha \left[  \frac{\partial^2 T}{\partial x^2} + \frac{\partial^2 T}{\partial y^2}   \right]
		\end{equation}

	\end{frame}





	\begin{frame}
		\frametitle{Modelo numérico explícito}

		$\bullet$ Todas as derivadas parciais espaciais de primeira ordem foram discretizadas com expansão em série de Taylor de primeira ordem em diferenças centradas.\\
		$\bullet$ Todas as derivadas parciais de segunda ordem foram discretizadas utilizando-se de expansão em série de Taylor de segunda ordem em diferenças centradas.\\
		$\bullet$ No tempo foi utilizado o método de Euler.\\

		\begin{equation}
			\begin{split}
			\frac{T_{i,j}^{k} - T_{i , j}^{k-1} }{\Delta t}
			= \alpha \left[  \frac{T_{i+1,j}^{k-1} - 2 T_{i,j}^{k-1} + T_{i-1,j}^{k-1} }{\Delta x^2} \right]\\
			+\alpha \left[\frac{T_{i,j+1}^{k-1} - 2 T_{i,j}^{k-1} + T_{i,j-1}^{k-1}}{\Delta y^2}\right] - u \frac{T_{i+1,j}^{k-1} - T_{i-1,j}^{k-1}}{2 \Delta x} - v \frac{T_{i,j+1}^{k-1} - T_{i , j-1}^{k-1}}{2 \Delta y}
			\end{split}
		\end{equation}

	\end{frame}





	\begin{frame}
		\frametitle{Modelo numérico explícito}
		$\bullet$ Com algumas simplificações chega-se na expressão utilizada no código:
		\begin{equation}
			\begin{split}
			T_{i,j}^{k} = T_{i,j}^{k-1} \left( 1 - 4 \frac{\alpha \Delta t}{\Delta s ^2}\right) + T_{i -1, j}^{k-1} \left( \alpha \frac{\Delta t}{\Delta s^2} + u \frac{\Delta t}{2 \Delta s} \right)\\
			+ T_{i,j-1}^{k-1} \left( \alpha \frac{\Delta t}{\Delta s^2} + v \frac{\Delta t}{2 \Delta s} \right) +  T_{i+1,j}^{k-1} \left( \alpha \frac{\Delta t}{ \Delta s^2} - u \frac{\Delta t}{2 \Delta s}\right) \\
			+  T_{i,j+1}^{k-1} \left( \alpha \frac{\Delta t}{\Delta s^2} - v \frac{\Delta t}{2 \Delta s}\right)
			\end{split}
		\end{equation}
	\end{frame}





	\begin{frame}
		\frametitle{Modelo numérico implícito}
		$\bullet$ Todas as derivadas parciais espaciais de primeira ordem foram discretizadas com expansão em série de Taylor de primeira ordem em diferenças centradas.\\
		$\bullet$ Todas as derivadas parciais de segunda ordem foram discretizadas utilizando-se expansão em série de Taylor de segunda ordem em diferenças centradas.\\
		$\bullet$ No tempo foi utilizado o método de Euler recuado de primeira ordem.\\
		\begin{equation}
			\begin{split}
			\frac{T_{i,j}^{k} - T_{i , j}^{k-1} }{\Delta t}
			= \alpha \left[  \frac{T_{i+1,j}^{k} - 2 T_{i,j}^{k} + T_{i-1,j}^{k} }{\Delta x^2} \right]\\
			+\alpha \left[\frac{T_{i,j+1}^{k} - 2 T_{i,j}^{k} + T_{i,j-1}^{k}}{\Delta y^2}\right] - u \frac{T_{i+1,j}^{k} - T_{i-1,j}^{k}}{2 \Delta x} - v \frac{T_{i,j+1}^{k} - T_{i , j-1}^{k}}{2 \Delta y}
			\end{split}
		\end{equation}
	\end{frame}





	\begin{frame}
		\frametitle{Modelo numérico implícito}
		$\bullet$ Com algumas simplificações chega-se na expressão utilizada no código:
		\begin{equation}
			\begin{split}
			T_{i,j}^{k} = \frac{T_{i,j}^{k-1} + T_{i -1, j}^{k} \left( \alpha \frac{\Delta t}{\Delta s^2} + u \frac{\Delta t}{2 \Delta s} \right) 	+ T_{i,j-1}^{k} \left( \alpha \frac{\Delta t}{\Delta s^2} + v \frac{\Delta t}{2 \Delta s} \right)}{ 1 - 4 \frac{\alpha \Delta t}{\Delta s ^2}} \\
			+ \frac{  T_{i+1,j}^{k} \left( \alpha \frac{\Delta t}{ \Delta s^2} - u \frac{\Delta t}{2 \Delta s}\right)
			+  T_{i,j+1}^{k} \left( \alpha \frac{\Delta t}{\Delta s^2} - v \frac{\Delta t}{2 \Delta s}\right)}{ 1 - 4 \frac{\alpha \Delta t}{\Delta s ^2}}
			\end{split}
		\end{equation}
	\end{frame}




	\begin{frame}
		\frametitle{Código em c++}
		\begin{minipage}[h!]{0.49\textwidth}
			$\bullet$ Foi traduzido do MatLab para c++ o código clássico e implementada as mudanças matemáticas deste novo caso.\\
			$\bullet$ Casos teste foram utilizados para validação qualitativa.\\
			$\bullet$ Alguns parâmetros de teste notáveis foram: \\
			- Alpha = 97 (Alumínio) \\
			- u = 5 m/s \\
			- v = 5 m/s \\
			$\bullet$ Para o explícito foi obedecida a condição de convergência:
			\begin{equation}
				\Delta t \leq \frac{\Delta s ^2}{4 \alpha}
			\end{equation}
		\end{minipage}
		\begin{minipage}[h!]{0.49\textwidth}
			\begin{figure}[h!]
				\centering
				\includegraphics[trim = {0cm 1cm 5cm 1cm}, clip , angle=0, scale=0.65]{images/printCodigo1}
				\caption{Código em c++ no editor.}
			\end{figure}
		\end{minipage}
	\end{frame}





	\begin{frame}
		\frametitle{OPENGL implementado em C++}
		O OPENGL para c++ é a implementação mais popular da API, junto da implementação em JAVA. Mas ela possui adaptações para várias linguagens, por meio de "bindings", ou seja, ligações do código às funções em c. A API fornece só uma forma de se comunicar com a placa de vídeo presente no computador. Não consiste em uma biblioteca, mas em um conjunto de definições de funções que já vem de fábrica instaladas nas placas de vídeo. Assim a API identifica a placa de vídeo e funciona como uma ponte entre o código e as funcionalidades nativas da placa de vídeo.
		\begin{minipage}[h!]{0.49\textwidth}
			\begin{figure}[h!]
				\centering
				\includegraphics[trim = {0cm 1cm 5cm 1cm}, clip , angle=0, scale=0.35]{images/cmaismaiscode}
				\caption{Chamada das definições OPENGL.}
			\end{figure}
		\end{minipage}
		\begin{minipage}[h!]{0.49\textwidth}
			\begin{figure}[h!]
				\centering
				\includegraphics[trim = {0cm 1cm 5cm 1cm}, clip , angle=0, scale=0.35]{images/criandojanela}
				\caption{Criação de janela com a biblioteca GLFW.}
			\end{figure}
			\begin{figure}[h!]
				\centering
				\includegraphics[trim = {0cm 0cm 0cm 0cm}, clip , angle=0, scale=0.35]{images/eventos}
				\caption{Função para processamento de evento GLFW.}
			\end{figure}

		\end{minipage}
	\end{frame}





	\begin{frame}
		\frametitle{Resultados iniciais}
		\begin{minipage}[h!]{0.49\textwidth}
			$\bullet$  Inicialmente, foi tratada somente condução térmica, as velocidades foram implementadas como zero, resultando na solução da equação clássica de balanço da energia térmica. Com a simulação de alguns casos, obteve-se os resultados ao lado.Apesar de se obedecer aos quesitos de convergência, não se consegue desenvolver uma simulação muito grande. E o resultado parece ser dependente da malha.
		\end{minipage}
		\begin{minipage}[h!]{0.49\textwidth}
			\begin{figure}[h!]
				\centering
				\includegraphics[trim = {1cm 1cm 1cm 1cm}, clip , angle=0, scale=0.45]{images/preliminar_results_1}
				\caption{Divergência da solução.}
			\end{figure}
		\end{minipage}
	\end{frame}





	\begin{frame}
		\frametitle{Resultados iniciais}
		\begin{minipage}[h!]{0.49\textwidth}
			\begin{figure}[h!]
				\centering
				\includegraphics[trim = {1cm 1cm 1cm 1cm}, clip , angle=0, scale=0.38]{images/preliminar_results_2}
				\caption{Simulação com poucos pontos.}
			\end{figure}
		\end{minipage}
		\begin{minipage}[h!]{0.49\textwidth}
			\begin{figure}[h!]
				\centering
				\includegraphics[trim = {1cm 1cm 1cm 1cm}, clip , angle=0, scale=0.45]{images/preliminar_results_3}
				\caption{Simulação com muitos pontos antes da divergência.}
			\end{figure}
		\end{minipage}
	\end{frame}





	\begin{frame}
		\frametitle{Resultados iniciais}
		$\bullet$ Mas, diminuindo-se o passo de tempo a valores muito pequenos, foi possível se alcançar a convergência para longos tempos de simulação, foi então que se observou um erro numérico na determinação do passo de tempo, fazendo a correção, obteve-se:\\
		\begin{minipage}[h!]{0.30\textwidth}
			\begin{figure}[h!]
				\centering
				\includegraphics[trim = {1cm 1cm 1cm 1cm}, clip , angle=0, scale=0.3]{images/sucesso_!}
				\caption{Condição inicial.}
			\end{figure}
		\end{minipage}
		\begin{minipage}[h!]{0.30\textwidth}
			\begin{figure}[h!]
				\centering
				\includegraphics[trim = {1cm 1cm 1cm 1cm}, clip , angle=0, scale=0.3]{images/sucesso_2}
				\caption{Situação intermediária.}
			\end{figure}
		\end{minipage}
		\begin{minipage}[h!]{0.30\textwidth}
			\begin{figure}[h!]
				\centering
				\includegraphics[trim = {1cm 1cm 1cm 1cm}, clip , angle=0, scale=0.3]{images/sucesso_3}
				\caption{Simulação bem sucedida.}
			\end{figure}
		\end{minipage}
	\end{frame}





	\begin{frame}
		\frametitle{Resultados iniciais}
		$\bullet$ Tendo-se sucesso no método clássico, acionou-se a velocidade a 5 m/s obtendo-se o seguinte resultado:\\
		\begin{minipage}[h!]{0.30\textwidth}
			\begin{figure}[h!]
				\centering
				\includegraphics[trim = {1cm 1cm 1cm 1cm}, clip , angle=0, scale=0.3]{images/sucesso_!}
				\caption{Condição inicial.}
			\end{figure}
		\end{minipage}
		\begin{minipage}[h!]{0.30\textwidth}
			\begin{figure}[h!]
				\centering
				\includegraphics[trim = {1cm 1cm 1cm 1cm}, clip , angle=0, scale=0.3]{images/sucesso_velocidade_2}
				\caption{Situação intermediária.}
			\end{figure}
		\end{minipage}
		\begin{minipage}[h!]{0.30\textwidth}
			\begin{figure}[h!]
				\centering
				\includegraphics[trim = {1cm 1cm 1cm 1cm}, clip , angle=0, scale=0.3]{images/sucesso_velocidade_3}
				\caption{Simulação bem sucedida.}
			\end{figure}
		\end{minipage}
	\end{frame}





	\begin{frame}
		\frametitle{Validação e análise de erros e convergência}
		\begin{minipage}[h!]{0.77\textwidth}
			$\bullet$ Com um programa funcional, chegou então a hora de se fazer uma análise de erros a fim de se validar quantitativamente o método. até mesmo com uma análise na ordem de convergência.  \\
			$\bullet$ Para isso, utilizou-se do artifício da solução manufaturada, que consiste em se criar um termo fonte e propor uma solução para T, de forma a se desenvolver uma solução exata para a equação quando acrescida deste termo fonte. Seguem a solução proposta e o termo fonte:
			\begin{equation}
				T = e ^{1 - (x^2 + y^2 + t)}
			\end{equation}
			\begin{equation}
				G = e^{1 - (x^2 + y^2 + t)} \left( 4 \alpha - 2 u x - 2 v y - 4 \alpha (x^2 + y^2) - 1 \right)
			\end{equation}
			$\bullet$ Assim temos a função analítica a ser analisada pelo método numérico como segue:
			\begin{equation}
				\frac{\partial T}{\partial t} + u \frac{\partial T}{\partial x} + v \frac{\partial T}{\partial y} = \alpha \left[  \frac{\partial^2 T}{\partial x^2} + \frac{\partial^2 T}{\partial y^2}   \right] + G
			\end{equation}
		\end{minipage}
		\begin{minipage}[h!]{0.17\textwidth}
			\begin{figure}[h!]
				\centering
				\includegraphics[trim = {0cm 0cm 0cm 0cm}, clip , angle=0, scale=0.1]{images/Analise_manufaturada}
				\caption{Resultado simulado.}
			\end{figure}
			\begin{figure}[h!]
				\centering
				\includegraphics[trim = {0cm 0cm 0cm 0cm}, clip , angle=0, scale=0.07]{images/resultado_analitico}
				\caption{ Resultado exato.}
			\end{figure}
		\end{minipage}
	\end{frame}





	\begin{frame}
		\frametitle{Modelo numérico explícito}
		$\bullet$ Com algumas simplificações chega-se na expressão utilizada no código:
		\begin{equation}
			\begin{split}
			T_{i,j}^{k} = T_{i,j}^{k-1} \left( 1 - 4 \frac{\alpha \Delta t}{\Delta s ^2}\right) + T_{i -1, j}^{k-1} \left( \alpha \frac{\Delta t}{\Delta s^2} + u \frac{\Delta t}{2 \Delta s} \right)\\
			+ T_{i,j-1}^{k-1} \left( \alpha \frac{\Delta t}{\Delta s^2} + v \frac{\Delta t}{2 \Delta s} \right) +  T_{i+1,j}^{k-1} \left( \alpha \frac{\Delta t}{ \Delta s^2} - u \frac{\Delta t}{2 \Delta s}\right) \\
			+  T_{i,j+1}^{k-1} \left( \alpha \frac{\Delta t}{\Delta s^2} - v \frac{\Delta t}{2 \Delta s}\right) + G
			\end{split}
		\end{equation}
	\end{frame}





	\begin{frame}
		\frametitle{Modelo numérico implícito}
		$\bullet$ Com algumas simplificações chega-se na expressão utilizada no código:
		\begin{equation}
			\begin{split}
			T_{i,j}^{k} = \frac{T_{i,j}^{k-1} + T_{i -1, j}^{k} \left( \alpha \frac{\Delta t}{\Delta s^2} + u \frac{\Delta t}{2 \Delta s} \right) 	+ T_{i,j-1}^{k} \left( \alpha \frac{\Delta t}{\Delta s^2} + v \frac{\Delta t}{2 \Delta s} \right)}{ 1 - 4 \frac{\alpha \Delta t}{\Delta s ^2}} \\
			+ \frac{  T_{i+1,j}^{k} \left( \alpha \frac{\Delta t}{ \Delta s^2} - u \frac{\Delta t}{2 \Delta s}\right)
			+  T_{i,j+1}^{k} \left( \alpha \frac{\Delta t}{\Delta s^2} - v \frac{\Delta t}{2 \Delta s}\right) + G}{ 1 - 4 \frac{\alpha \Delta t}{\Delta s ^2}}
			\end{split}
		\end{equation}
	\end{frame}





	\begin{frame}
		\frametitle{Resultados das simulações}
		$\bullet$ Dessa forma, foram conduzidas várias simulações em diferentes CFL`s, de forma a se observar o comportamento do erro a partir destes desenvolvimentos, seguem os resultados: \\
		\begin{minipage}[h!]{0.49\textwidth}
			\begin{figure}[h!]
				\centering
				\includegraphics[trim = {0cm 0cm 0cm 0cm}, clip , angle=0, scale=0.4]{images/analise_de_erros_explicito}
				\caption{ Análise de convergência para o método explicito. Uma análise das normas revelou uma convergência de 2 ordem. Quando se dobrava o numero de células, se dividia por aproximadamente 4 o erro.}
			\end{figure}
		\end{minipage}
		\begin{minipage}[h!]{0.49\textwidth}
			\begin{figure}[h!]
				\centering
				\includegraphics[trim = {0cm 0cm 0cm 0cm}, clip , angle=0, scale=0.4]{images/analise_de_erros_implicito}
				\caption{ Análise de convergência para o método implícito. Uma análise das normas revelou uma convergência de 2 ordem. Quando se dobrava o numero de células, se dividia por aproximadamente 4 o erro.}
			\end{figure}
		\end{minipage}
	\end{frame}





	\begin{frame}
		\frametitle{Código em FORTRAN}
		\begin{minipage}[h!]{0.49\textwidth}

			Diferente do C++, onde se usa a biblioteca GLFW para criação de janelas, no FORTRAN se utiliza o GLUT. Não foi possível localizar bidings para GLFW já prontas. E decidiu-se que seria mais simples aprender a nova biblioteca.

			$\bullet$ Foi traduzido do c++ para FORTRAN o código de condução com advecção.\\
			$\bullet$ Casos teste foram utilizados para validação qualitativa.\\
			$\bullet$ Alguns parâmetros de teste notáveis foram: \\
			- Alpha = 97 (Alumínio) \\
			- u = 5 m/s \\
			- v = 5 m/s \\
			$\bullet$ Para o explícito foi obedecida a condição de convergência:
			\begin{equation}
			\Delta t \leq \frac{\Delta s ^2}{4 \alpha}
			\end{equation}
		\end{minipage}
		\begin{minipage}[h!]{0.49\textwidth}
			\begin{figure}[h!]
				\centering
				\includegraphics[trim = {0cm 1cm 5cm 1cm}, clip , angle=0, scale=0.45]{images/fortran_code}
				\caption{Código em FORTRAN com implementação de MPI e OPENGL.}
			\end{figure}
		\end{minipage}
	\end{frame}


\tikzstyle{rect} = [draw, rectangle,fill = red!20, text width=6em, text centered, minimum height = 2em ]
\tikzstyle{elli} = [draw, ellipse,fill=white!20,minimum height=2em]
\tikzstyle{circ} = [draw, circle,fill=white!20, minimum width=8pt,inner sep=10pt]
\tikzstyle{diam} = [draw, diamond,fill=white!20,text width=6em, text badly centered, inner sep=0pt]
\tikzstyle{line} = [draw, -latex']


	\begin{frame}
		\frametitle{Uso do MPI no código}
		\begin{minipage}[h!]{0.35\textwidth}
			Criam-se dois processos em paralelo. Um dedicado à renderização dos resultados e o outro dedicado somente à simulação.\\
			A forma encontrada de sincronizar ambos os processos foi dispondo funções de envio e recebimento sempre complementares, garantindo mensagens nos dois sentidos para cada mensagem trocada.
		\end{minipage}	\hspace{0.5cm}
		\begin{minipage}[h!]{0.59\textwidth}
			\begin{figure}[h]
				\begin{center}
					\begin{tikzpicture}[node distance = 1.5cm, auto]
					\node[rect, rounded corners](step1) {$ Source Code $};
					\node[rect,rounded corners , below of=step1, node distance=2.5cm] (step2) {$ IF(RANK) $};
					\node[rect, right of=step2, node distance=3.5cm] (step3) {$Simulation()$};
					\node[rect, left of=step2, node distance=3.5cm] (step4){$Vizualization()$};

					\path[line](step1) -- node [right, text width=4em] {mpirun -n 2.} node [left, text width=4em] {} (step2);

					\path[line](step2) -- node [above,near end, text width=4em] { = 0} node (line2) [below, text width=4em] {} (step3);

					\path[line](step2) -- node [above,near start, text width=4em] { = 1} node [below, text width=4em] {} (step4);

					\path[line](step3) -- ++(0,-1) -- ++(-7,0) -- node[xshift= 5cm,yshift= -0.7cm, text width=2.5cm]{MPI\_SEND()}(step4);

					\end{tikzpicture}
				\end{center}
				\caption{Fluxograma do uso de MPI no algorítmo.}
			\end{figure}

		Após a criação dos processos filhos, tem-se a comunicação entre eles por MPI\_SEND e MPI\_RCEV.
		\end{minipage}
	\end{frame}





	\begin{frame}
		\frametitle{Resultados}

		\begin{minipage}[h!]{0.45\textwidth}
	 		Com o devido cuidado com o código foi possível se obter o mesmo resultado do c++ com o fortran:
	 	\end{minipage}\hfill
 		\begin{minipage}[h!]{0.5\textwidth}
	 		\begin{figure}
	 			\animategraphics[scale = 0.25 , loop , autoplay]{10}{images/fortopengl/frame-}{0}{25}
	 			\label{gif_gfort}
	 			\caption{Código em FORTRAN executando OPANGL e MPI.}
	 		\end{figure}
	 	\end{minipage}
	\end{frame}




%%%%%%%%%%%%%%%%%%%%%%%%%%%%%%%%%%%%%%%%%%%%%%%%%%%%%%%%%%%%%%%%%%%%%%%%%%%%%%%%%%%%%%%%%%%%%%
\section{Desenvolvimento matemático dinâmico}

	\begin{frame}
		\frametitle{Equações representativas}
		\flushleft
		Para este desenvolvimento matemático serão utilizadas:\\
		\quad $\bullet$ As equações de Navier Stokes.\\
		\quad $\bullet$ A equação da continuidade.\\
		\quad $\bullet$ A equação de balanço da energia térmica.\\
		\quad $\bullet$ Uma equação de estado.\\
		\quad $\bullet$ Aproximação de Boussinesq.\\

		\begin{equation}
		\nabla . \vec{V} = 0
		\end{equation}
		\begin{equation}
		\frac{\partial \vec{V}}{\partial t} +  \vec{V} . {\nabla} \vec{V}  =  -\frac{1}{\rho_o} {\nabla}P + \frac{\rho - \rho_o}{\rho_o} \vec{g} + \nu \nabla ^2 \vec{V}
		\end{equation}
		\begin{equation}
		\frac{\partial T}{\partial t} + \vec{\nabla} . \left( \vec{V}T \right) = \alpha \nabla^2T
		\end{equation}
		\begin{equation}
		\frac{\rho - \rho_o}{\rho_o} = \beta \left( T - T_o\right)
		\end{equation}

	\end{frame}





	\begin{frame}
		\frametitle{Desenvolvendo pressão e velocidade no modelo explícito}
		\flushleft
		Primeiramente se desenvolvem as equações de Navier-Stokes de duas formas:\vspace{0.5cm}
		\centering
		\underline{A partir do domínio da pressão no passo posterior:}
		\begin{equation}\label{equation1}
		\frac{\vec{V}^{n + 1} - \vec{V}^{n}}{\Delta t} + \vec{V}^{n} . {\nabla} \vec{V}^{n} = - \frac{1}{\rho_o}\nabla P^{n + 1} + \left( \frac{\rho - \rho_o}{\rho_o} \right) \vec{g} + \nu \nabla^2 \vec{V}^{n}
		\end{equation}
		\underline{A partir do domínio da pressão no passo atual:}
		\begin{equation}\label{eqlabel1}
		\frac{\vec{V}^{\ast{n + 1}} - \vec{V}^{n}}{\Delta t} + \vec{V}^{n} . {\nabla} \vec{V}^{n} = - \frac{1}{\rho_o}\nabla P^{n} + \left( \frac{\rho - \rho_o}{\rho_o} \right) \vec{g} + \nu \nabla^2 \vec{V}^{n}
		\end{equation}

		Assim, subtraindo-se as duas, tem-se uma relação que descreve a relação entre os valores do domínio da pressão:

		\begin{equation}
		\frac{\vec{V}^{{n + 1}} - \vec{V}^{{\ast n+1} }}{\Delta t} = - \frac{1}{\rho_o}\nabla \left( P^{n+1} - P ^n\right)
		\end{equation}



	\end{frame}




	\begin{frame}
		\frametitle{Diferença de pressão}
		\flushleft
		Assim, descreve-se uma nova variável $ P^\prime $ que denota a diferença na pressão entre os passos, como:

		\begin{equation}
		P^\prime = P^{n + 1} - P^n
		\end{equation}
		Assim temos:

		\begin{equation}\label{eqlabel3}
		\vec{V}^{n+1} - \vec{V}^{\ast{n + 1}} = - \frac{\Delta t}{\rho_o} \nabla P^\prime
		\end{equation}

		Aplicando-se o divergente em ambos os lados temos:

		\begin{equation}
		\nabla . \vec{V}^{n+1} - \nabla .\vec{V}^{\ast{n + 1}} = - \frac{\Delta t}{\rho_o} \nabla^2 P^\prime
		\end{equation}

		A partir da equação da continuidade temos:

		\begin{equation}\label{eqlabel2}
		\nabla .\vec{V}^{\ast{n + 1}} = \frac{\Delta t}{\rho_o} \nabla^2 P^\prime
		\end{equation}


	\end{frame}


\tikzstyle{rect} = [draw, rectangle,fill = red!20, text width=5em, text centered, minimum height = 2em ]
\tikzstyle{elli} = [draw, ellipse,fill=white!20,minimum height=2em]
\tikzstyle{circ} = [draw, circle,fill=white!20, minimum width=8pt,inner sep=10pt]
\tikzstyle{diam} = [draw, diamond,fill=white!20,text width=6em, text badly centered, inner sep=0pt]
\tikzstyle{line} = [draw, -latex']

	\begin{frame}
		\frametitle{Fluxograma de dados por iteração}
		\flushleft
		Assim tem-se um valor inicial de velocidade estimada, que é utilizado para corrigir o domínio de pressão.
		\begin{figure}[h]
			\begin{center}
				\begin{tikzpicture}[node distance = 1.5cm, auto]
					\node[rect, rounded corners](step1) {$\vec{V}^n $};
					\node[rect,rounded corners , right of=step1, node distance=4.5cm] (step2) {$\vec{V}^{* n + 1} $};
					\node[rect, right of=step2, node distance=6.5cm] (step3) {$P^\prime$};
					\node[rect, rounded corners, below of=step3, node distance=2.5cm](step5){$ \vec{V}^{n+1}$};

					\path[line](step1) -- node [above, text width=4em] {Resolução explicita.} node [below, text width=4em] {Eq.\ref{eqlabel1}} (step2);

					\path[line](step2) -- node [above, text width=4em] {Resolução implicita.} node (line2) [below, text width=4em] {Eq.\ref{eqlabel2}} (step3);

					\path[line](step3) -- node [right, near start,  text width=4em] {Resolução dinâmica.} node [right, near end, text width=4em] {Eq.\ref{eqlabel3}} (step5);

					\path[line](step5) -| node [right, near end, text width=6em] {Atualiza\\domínio\\dinâmico.} node [below, near start, text width=12em] {Checa continuidade.} (step1);
				\end{tikzpicture}
			\end{center}
		\caption{Fluxograma do loop principal do algoritmo.}
		\end{figure}
	\end{frame}





	\begin{frame}
		\frametitle{Desenvolvimento das equações de Navier-Stokes}
		\centering
		\begin{equation}
			\frac{\partial \vec{V}}{\partial t} +  \vec{V} . {\nabla} \vec{V}  =  -\frac{1}{\rho_o} {\nabla}P + \frac{\rho - \rho_o}{\rho_o} \vec{g} + \nu \nabla ^2 \vec{V}
		\end{equation}
		Com as equações de Navier-Stokes no formato vetorial, separam-se nas três componentes:

		\begin{equation}
			\frac{\partial u}{\partial t} + u\frac{\partial u}{\partial x} + v\frac{\partial u}{\partial y} + w\frac{\partial u }{\partial z}  =  -\frac{1}{\rho_o} \frac{\partial P}{\partial x} + \frac{\rho - \rho_o}{\rho_o} g_x + \nu \left( \frac{\partial ^2 u}{\partial x^2} + \frac{\partial ^2 u}{\partial y^2} + \frac{\partial ^2 u}{\partial z^2} \right)
		\end{equation}

		\begin{equation}
			\frac{\partial v}{\partial t} + u\frac{\partial v}{\partial x} + v\frac{\partial v}{\partial y} + w\frac{\partial v }{\partial z}  =  -\frac{1}{\rho_o} \frac{\partial P}{\partial y} + \frac{\rho - \rho_o}{\rho_o} g_y + \nu \left( \frac{\partial ^2 v}{\partial x^2} + \frac{\partial ^2 v}{\partial y^2} + \frac{\partial ^2 v}{\partial z^2} \right)
		\end{equation}

		\begin{equation}
			\frac{\partial w}{\partial t} + u\frac{\partial w}{\partial x} + v\frac{\partial w}{\partial y} + w\frac{\partial w }{\partial z}  =  -\frac{1}{\rho_o} \frac{\partial P}{\partial z} + \frac{\rho - \rho_o}{\rho_o} g_z + \nu \left( \frac{\partial ^2 w}{\partial x^2} + \frac{\partial ^2 w}{\partial y^2} + \frac{\partial ^2 w}{\partial z^2} \right)
		\end{equation}
	\end{frame}





	\begin{frame}
		\frametitle{Desenvolvimento das equações de Navier-Stokes}
		\centering
		\begin{equation}
			\frac{\partial u}{\partial t} + u\frac{\partial u}{\partial x} + v\frac{\partial u}{\partial y} + {\color{red} w\frac{\partial u }{\partial z} } =  -\frac{1}{\rho_o} \frac{\partial P}{\partial x} + \frac{\rho - \rho_o}{\rho_o} g_x + \nu \left( \frac{\partial ^2 u}{\partial x^2} + \frac{\partial ^2 u}{\partial y^2} + { \color{red} \frac{\partial ^2 u}{\partial z^2} } \right)
		\end{equation}

		\begin{equation}
			\frac{\partial v}{\partial t} + u\frac{\partial v}{\partial x} + v\frac{\partial v}{\partial y} + {\color{red} w\frac{\partial v }{\partial z} } =  -\frac{1}{\rho_o} \frac{\partial P}{\partial y} + \frac{\rho - \rho_o}{\rho_o} g_y + \nu \left( \frac{\partial ^2 v}{\partial x^2} + \frac{\partial ^2 v}{\partial y^2} + { \color{red} \frac{\partial ^2 v}{\partial z^2} } \right)
		\end{equation}

		{\color{red}
			\begin{equation}
				\frac{\partial w}{\partial t} + u\frac{\partial w}{\partial x} + v\frac{\partial w}{\partial y} + w\frac{\partial w }{\partial z}  =  -\frac{1}{\rho_o} \frac{\partial P}{\partial z} + \frac{\rho - \rho_o}{\rho_o} g_z + \nu \left( \frac{\partial ^2 w}{\partial x^2} + \frac{\partial ^2 w}{\partial y^2} + \frac{\partial ^2 w}{\partial z^2} \right)
			\end{equation}
		}

		Tirando os termos referentes à dimensão z, a partir de considerações de autosimilaridade.


	\end{frame}





	\begin{frame}
		\frametitle{Desenvolvimento das equações de Navier-Stokes de forma explícita}
		\centering
		\begin{equation}
			\frac{\partial u}{\partial t} + u\frac{\partial u}{\partial x} + v\frac{\partial u}{\partial y} =  -\frac{1}{\rho_o} \frac{\partial P}{\partial x} + \frac{\rho - 	\rho_o}{\rho_o} g_x + \nu \left( \frac{\partial ^2 u}{\partial x^2} + \frac{\partial ^2 u}{\partial y^2} \right)
		\end{equation}

		\begin{equation}
			\frac{\partial v}{\partial t} + u\frac{\partial v}{\partial x} + v\frac{\partial v}{\partial y} =  -\frac{1}{\rho_o} \frac{\partial P}{\partial y} + \frac{\rho - \rho_o}{\rho_o} g_y + \nu \left( \frac{\partial ^2 v}{\partial x^2} + \frac{\partial ^2 v}{\partial y^2} \right)
		\end{equation}
		Discretizando estes termos, a partir de expansões em série de Taylor, tem-se:
		\begin{equation}
			\begin{split}
			\frac{u_{i , j}^{\ast k + 1} - u_{i , j}^{k}}{\Delta t} + u_{i , j}^{k}\frac{u_{i + 1 , j}^k - u_{i - 1 , j}^k  }{2 \Delta x} + v_{i , j}^{k}\frac{u_{i , j+ 1}^k - u_{i, j-1}^k  }{2 \Delta y} =  -\frac{1}{\rho_o} \frac{P_{i, j}^k - P_{i - 1, j}^k}{\Delta x} \\ + \frac{\rho - 	\rho_o}{\rho_o} g_x + \nu \left( \frac{u_{i+1 , j}^{k} - 2 u_{i,j}^{k} + u_{i-1,j}^{k}}{\Delta x^2} + \frac{u_{i , j+1}^{k} - 2 u_{i,j}^{k} + u_{i,j-1}^{ k}}{\Delta y^2} \right)
			\end{split}
		\end{equation}


	\end{frame}




	\begin{frame}
		\frametitle{Desenvolvimento das equações de Navier-Stokes  de forma explícita}
		Dessa forma tem-se as duas equações de balanço de quantidade de movimento linear discretizadas:
		\begin{equation}
			\begin{split}
			\frac{u_{i , j}^{\ast k + 1} - u_{i , j}^{k}}{\Delta t} + u_{i , j}^{k}\frac{u_{i + 1 , j}^k - u_{i - 1 , j}^k  }{2 \Delta x} + v_{i , j}^{k}\frac{u_{i , j+ 1}^k - u_{i, j-1}^k  }{2 \Delta y} =  -\frac{1}{\rho_o} \frac{P_{i, j}^k - P_{i - 1 , j}^k}{\Delta x} \\ + \frac{\rho - 	\rho_o}{\rho_o} g_x + \nu \left( \frac{u_{i+1 , j}^{k} - 2 u_{i,j}^{k} + u_{i-1,j}^{k}}{\Delta x^2} + \frac{u_{i , j+1}^{k} - 2 u_{i,j}^{k} + u_{i,j-1}^{k}}{\Delta y^2} \right)
			\end{split}
		\end{equation}
		\begin{equation}
			\begin{split}
			\frac{v_{i , j}^{\ast k + 1} - v_{i , j}^{k}}{\Delta t} + u_{i , j}^{k}\frac{v_{i + 1 , j}^k - v_{i - 1 , j}^k  }{2 \Delta x} + v_{i , j}^{k}\frac{v_{i , j+ 1}^k - v_{i, j-1}^k  }{2 \Delta y} =  -\frac{1}{\rho_o} \frac{P_{i , j}^k - P_{i , j - 1}^k}{\Delta y} \\ + \frac{\rho - \rho_o}{\rho_o} g_y + \nu \left( \frac{v_{i+1 , j}^{k} - 2 v_{i,j}^{k} + v_{i-1,j}^{k}}{\Delta x^2} + \frac{v_{i , j+1}^{k} - 2 v_{i,j}^{k} + v_{i,j-1}^{k}}{\Delta y^2} \right)
			\end{split}
		\end{equation}
	\end{frame}




	\begin{frame}
		\frametitle{Desenvolvimento das equações de Navier-Stokes  de forma explícita}
		Assim, isolando-se o termo de avanço, tem-se o método explícito:
		\begin{equation}
			\begin{split}
			u_{i , j}^{\ast k + 1} = u_{i , j}^{k} - u_{i , j}^{k}\frac{\Delta t}{2 \Delta x} \left( u_{i + 1 , j}^k - u_{i - 1 , j}^k  \right) - v_{i , j}^{k}\frac{\Delta t }{2 \Delta y} \left(u_{i , j+ 1}^k - u_{i, j-1}^k \right)  -\frac{\Delta t}{\rho_o \Delta x} \left(P_{i, j}^k - P_{i - 1 , j}^k\right)\\ + \Delta t  g_x  \frac{\rho - \rho_o}{\rho_o} + \frac{\nu \Delta t}{\Delta x^2} \left( u_{i+1 , j}^{k} - 2 u_{i,j}^{k} + u_{i-1,j}^{k} \right) + \frac{\nu \Delta t}{\Delta y^2} \left(u_{i , j+1}^{k} - 2 u_{i,j}^{k} + u_{i,j-1}^{k}\right)
			\end{split}
		\end{equation}
		\begin{equation}
			\begin{split}
			v_{i , j}^{\ast k + 1} = v_{i , j}^{k} -u_{i , j}^{k}\frac{\Delta t}{2 \Delta x} \left(v_{i + 1 , j}^k - v_{i - 1 , j}^k\right) - v_{i , j}^{k} \frac{\Delta t}{2 \Delta y} \left(v_{i , j+ 1}^k - v_{i, j-1}^k  \right) -\frac{\Delta t}{\rho_o \Delta y} \left(P_{i , j}^k - P_{i , j - 1}^k\right) \\ + \Delta t g_y  \frac{\rho - \rho_o}{\rho_o} + \frac{\nu \Delta t}{\Delta x^2} \left( v_{i+1 , j}^{k} - 2 v_{i,j}^{k} + v_{i-1,j}^{k} \right) + \frac{\nu \Delta t}{\Delta y^2} \left( v_{i , j+1}^{k} - 2 v_{i,j}^{k} + v_{i,j-1}^{k} \right)
			\end{split}
		\end{equation}
	\end{frame}






	\begin{frame}
		\frametitle{Desenvolvimento das equações de Navier-Stokes  de forma implicita}
		Dessa forma tem-se as duas equações de balanço de quantidade de movimento linear discretizadas:
		\begin{equation}
			\begin{split}
			\frac{u_{i , j}^{\ast k + 1} - u_{i , j}^{k}}{\Delta t} + u_{i , j}^{k}\frac{u_{i + 1 , j}^k - u_{i - 1 , j}^k  }{2 \Delta x} + v_{i , j}^{k}\frac{u_{i , j+ 1}^k - u_{i, j-1}^k  }{2 \Delta y} =  -\frac{1}{\rho_o} \frac{P_{i, j}^k - P_{i - 1 , j}^k}{\Delta x} \\ + \frac{\rho - 	\rho_o}{\rho_o} g_x + \nu \left( \frac{u_{i+1 , j}^{\ast k + 1} - 2 u_{i,j}^{\ast k + 1} + u_{i-1,j}^{\ast k + 1}}{\Delta x^2} + \frac{u_{i , j+1}^{\ast k + 1} - 2 u_{i,j}^{\ast k + 1} + u_{i,j-1}^{\ast k + 1}}{\Delta y^2} \right)
			\end{split}
		\end{equation}
		\begin{equation}
			\begin{split}
			\frac{v_{i , j}^{\ast k + 1} - v_{i , j}^{k}}{\Delta t} + u_{i , j}^{k}\frac{v_{i + 1 , j}^k - v_{i - 1 , j}^k  }{2 \Delta x} + v_{i , j}^{k}\frac{v_{i , j+ 1}^k - v_{i, j-1}^k  }{2 \Delta y} =  -\frac{1}{\rho_o} \frac{P_{i , j}^k - P_{i , j - 1}^k}{\Delta y} \\ + \frac{\rho - \rho_o}{\rho_o} g_y + \nu \left( \frac{v_{i+1 , j}^{\ast k + 1} - 2 v_{i,j}^{\ast k + 1} + v_{i-1,j}^{\ast k + 1}}{\Delta x^2} + \frac{v_{i , j+1}^{\ast k + 1} - 2 v_{i,j}^{\ast k + 1} + v_{i,j-1}^{\ast k + 1}}{\Delta y^2} \right)
			\end{split}
		\end{equation}
	\end{frame}





	\begin{frame}
		\frametitle{Desenvolvimento das equações de Navier-Stokes  de forma implicita}
		\begin{equation}
			\begin{split}
			u_{i , j}^{\ast k + 1} = \frac{u_{i , j}^{k} \Delta x ^2 \Delta y^2}{ \Delta x ^2 \Delta y^2 + 2 \nu \Delta t(\Delta x ^2 + \Delta y^2)} - \frac{ v_{i , j}^{k} \Delta x ^2 \Delta y \Delta t}{ 2 \Delta x ^2 \Delta y^2 + 4 \nu \Delta t(\Delta x ^2 + \Delta y^2)} \left( u_{i , j+ 1}^k - u_{i, j-1}^k   \right)  \\ - \frac{u_{i , j}^{k} \Delta x \Delta y^2 \Delta t}{ 2 \Delta x ^2 \Delta y^2 + 4 \nu \Delta t(\Delta x ^2 + \Delta y^2)} \left(u_{i + 1 , j}^k - u_{i - 1 , j}^k \right) - \frac{ \Delta x \Delta y^2 \Delta t}{ \rho_o \Delta x ^2 \Delta y^2 + 2 \rho_o \nu \Delta t(\Delta x ^2 + \Delta y^2)} (P_{i, j}^k - P_{i - 1 , j}^k ) \\ + \frac{ \Delta x ^2 \Delta y^2 \Delta t (\rho -\rho_o) g_x }{ \rho_o \Delta x ^2 \Delta y^2 + 2 \rho_o \nu \Delta t(\Delta x ^2 + \Delta y^2)} + \frac{ \nu \Delta y^2 {\Delta t}}{ \Delta x ^2 \Delta y^2 + 2 \nu \Delta t(\Delta x ^2 + \Delta y^2)} \left( u_{i+1 , j}^{\ast k + 1} + u_{i-1,j}^{\ast k + 1} \right) \\ + \frac{ \nu \Delta x^2 {\Delta t}}{ \Delta x ^2 \Delta y^2 + 2 \nu \Delta t(\Delta x ^2 + \Delta y^2)} \left( u_{i , j+1}^{\ast k + 1} + u_{i,j-1}^{\ast k + 1}\right)
			\end{split}
		\end{equation}
	
	\end{frame}




	\begin{frame}
		\frametitle{Desenvolvimento das equações de Navier-Stokes  de forma implicita}
	
		\begin{equation}
			\begin{split}
			v_{i , j}^{\ast k + 1} = \frac{v_{i , j}^{k} \Delta x ^2 \Delta y^2}{ \Delta x ^2 \Delta y^2 + 2 \nu \Delta t(\Delta x ^2 + \Delta y^2)} - \frac{ v_{i , j}^{k} \Delta x ^2 \Delta y \Delta t}{ 2 \Delta x ^2 \Delta y^2 + 4 \nu \Delta t(\Delta x ^2 + \Delta y^2)} \left( v_{i , j+ 1}^k - v_{i, j-1}^k   \right)  \\ - \frac{u_{i , j}^{k} \Delta x \Delta y^2 \Delta t}{ 2 \Delta x ^2 \Delta y^2 + 4 \nu \Delta t(\Delta x ^2 + \Delta y^2)} \left(v_{i + 1 , j}^k - v_{i - 1 , j}^k \right) - \frac{ \Delta x^2 \Delta y \Delta t}{ \rho_o \Delta x ^2 \Delta y^2 + 2 \rho_o \nu \Delta t(\Delta x ^2 + \Delta y^2)} (P_{i, j}^k - P_{i , j-1}^k ) \\ + \frac{ \Delta x ^2 \Delta y^2 \Delta t (\rho -\rho_o) g_y }{ \rho_o \Delta x ^2 \Delta y^2 + 2 \rho_o \nu \Delta t(\Delta x ^2 + \Delta y^2)} + \frac{ \nu \Delta y^2 {\Delta t}}{ \Delta x ^2 \Delta y^2 + 2 \nu \Delta t(\Delta x ^2 + \Delta y^2)} \left( v_{i+1 , j}^{\ast k + 1} + v_{i-1,j}^{\ast k + 1} \right) \\ + \frac{ \nu \Delta x^2 {\Delta t}}{ \Delta x ^2 \Delta y^2 + 2 \nu \Delta t(\Delta x ^2 + \Delta y^2)} \left( v_{i , j+1}^{\ast k + 1} + v_{i,j-1}^{\ast k + 1}\right)
			\end{split}
		\end{equation}
	
	\end{frame}






	\begin{frame}
		\frametitle{Efetuando a correção na pressão}
		Resgatando a equação para correção da pressão, tem-se:
		\begin{equation}
			\nabla .\vec{V}^{\ast{n + 1}} = \frac{\Delta t}{\rho_o} \nabla^2 P^\prime
		\end{equation}
		Discretizando a equação:
		\begin{equation}
			\frac{u_{i + 1 , j}^{\ast k + 1} - u_{i , j}^{\ast k + 1}}{\Delta x} + \frac{v_{i , j + 1}^{\ast k + 1} - v_{i , j}^{\ast k + 1}}{\Delta y} = \frac{\Delta t}{\rho_o} * \left( \frac{ P^{\prime}_{i + 1, j} - 2 P^{\prime}_{i , j} + P^{\prime}_{i - 1 , j} }{\Delta x^2} + \frac{ P^{\prime}_{i, j + 1} - 2 P^{\prime}_{i , j} + P^{\prime}_{i , j- 1} }{\Delta y^2} \right)
		\end{equation}
		\begin{equation}
		\frac{\rho_o}{ \Delta t} * \frac{u_{i + 1 , j}^{\ast k + 1} - u_{i , j}^{\ast k + 1}}{\Delta x} + \frac{\rho_o}{ \Delta t} * \frac{v_{i , j + 1}^{\ast k + 1} - v_{i , j}^{\ast k + 1}}{\Delta y} =  \frac{ P^{\prime}_{i + 1, j} - 2 P^{\prime}_{i , j} + P^{\prime}_{i - 1 , j} }{\Delta x^2} + \frac{ P^{\prime}_{i, j + 1} - 2 P^{\prime}_{i , j} + P^{\prime}_{i , j- 1} }{\Delta y^2}
		\end{equation}



	\end{frame}



	\begin{frame}
		\frametitle{Efetuando a correção na pressão}
		Assim, tem-se a expressão utilizada no código:
		\begin{equation}
			\frac{ P^{\prime}_{i + 1, j} - 2 P^{\prime}_{i , j} + P^{\prime}_{i - 1 , j} }{\Delta x^2} + \frac{ P^{\prime}_{i, j + 1} - 2 P^{\prime}_{i , j} + P^{\prime}_{i , j- 1} }{\Delta y^2} = \frac{\rho_o}{ \Delta t \Delta x} \left(u_{i + 1 , j}^{\ast k + 1} - u_{i , j}^{\ast k + 1} \right) + \frac{\rho_o}{ \Delta t \Delta y } \left( v_{i , j + 1}^{\ast k + 1} - v_{i , j}^{\ast k + 1}\right)
		\end{equation}

		\begin{equation}
			\begin{split}
			-2 P^{\prime}_{i , j} \left(   \frac{\Delta y^2 + \Delta x^2}{\Delta x^2 \Delta y^2} \right)  = \frac{\rho_o}{ \Delta t \Delta x} \left(u_{i + 1 , j}^{\ast k + 1} - u_{i , j}^{\ast k + 1} \right) + \\ \frac{\rho_o}{ \Delta t \Delta y } \left( v_{i , j + 1}^{\ast k + 1} - v_{i , j}^{\ast k + 1}\right) - \frac{1}{\Delta y^2} \left(P^{\prime}_{i , j- 1} + P^{\prime}_{i, j + 1} \right) - \frac{1}{\Delta x^2} \left(P^{\prime}_{i + 1, j} +  P^{\prime}_{i - 1 , j}\right)
			\end{split}
		\end{equation}

		\begin{equation}
		\begin{split}
		P^{\prime}_{i , j}  = - \frac{\rho_o \Delta x \Delta y^2 }{ 2 ( \Delta y^2 + \Delta x^2) \Delta t} \left(u_{i + 1 , j}^{\ast k + 1} - u_{i , j}^{\ast k + 1} \right) + \frac{\Delta y^2}{ 2 (\Delta y^2 + \Delta x^2)} \left(P^{\prime}_{i + 1, j} +  P^{\prime}_{i - 1 , j}\right) \\ - \frac{\rho_o \Delta x^2 \Delta y }{ 2 ( \Delta y^2 + \Delta x^2) \Delta t} \left( v_{i , j + 1}^{\ast k + 1} - v_{i , j}^{\ast k + 1}\right) + \frac{\Delta x^2}{ 2 (\Delta y^2 + \Delta x^2)} \left(P^{\prime}_{i , j- 1} + P^{\prime}_{i, j + 1} \right)
		\end{split}
		\end{equation}

	\end{frame}


	\begin{frame}
		\frametitle{Efetuando a correção na pressão}
		A partir da matriz de $p^\prime$ é possível se chegar no próximo passo real da velocidade:
		\begin{equation}
			\vec{V}^{k+1}  = \vec{V}^{\ast{k + 1}} - \frac{\Delta t}{\rho_o} \nabla P^\prime
		\end{equation}
		Separando-se nas componentes:
		\begin{equation}
			u^{k+1}_{i, j}  = u^{\ast k+1}_{i, j} - \frac{\Delta t}{\Delta x \rho_o} \left(  P^{\prime}_{i , ii} -   P^{\prime}_{i - 1, ii}  \right)
		\end{equation}

		\begin{equation}
		v^{k+1}_{i, j}  = v^{\ast k+1}_{i, j} - \frac{\Delta t}{\Delta y \rho_o}  \left(  P^{\prime}_{i , ii} -   P^{\prime}_{i, ii-1}    \right)
		\end{equation}

		Dessa forma, já é possível se desenvolver o domínio dinâmico segundo as equações de Navier-Stokes.


	\end{frame}




	\begin{frame}
		\frametitle{Efetuando a correção na pressão}
		A partir desta etapa, basta checar se o divergente do campo de velocidades está próximo de zero, de forma a se obedecer a equação da continuidade:
		\begin{equation}
		\frac{u_{i + 1 , ii}^{k + 1} - u_{i, ii}^{k + 1} }{\Delta x} + \frac{v_{i , ii + 1}^{k + 1} - v_{i, ii}^{k + 1} }{\Delta y} \simeq 0
		\end{equation}

	\end{frame}

	\begin{frame}
		\frametitle{Resultados parciais}
		Quando colocado estas equações no código, houveram os seguintes resultados:\\
		\vspace{1cm}
		\begin{minipage}[h!]{0.31\textwidth}
			\begin{figure}
				\animategraphics[scale = 0.2 , loop , autoplay]{10}{images/temperature_1/frame-}{3}{120}
				\label{gif1}
				\caption{Velocidade vertical.}
			\end{figure}
		\end{minipage}
		\begin{minipage}[h!]{0.31\textwidth}
			\begin{figure}
				\animategraphics[scale = 0.2 , loop , autoplay]{10}{images/u_1/frame-}{5}{120}
				\label{gif2}
				\caption{Velocidade horizontal.}
			\end{figure}
		\end{minipage}
		\begin{minipage}[h!]{0.31\textwidth}
			\begin{figure}
				\animategraphics[scale = 0.2 , loop , autoplay]{10}{images/P_1/frame-}{5}{120}
				\label{gif3}
				\caption{Pressão.}
			\end{figure}
		\end{minipage}
	
	Houveram problemas, após uma revisão cuidados encontrou-se um erro nas equações dinâmicas.

	\end{frame}




	\begin{frame}
		\frametitle{Resultados parciais}
		\vspace{1cm}
		\begin{minipage}[h!]{0.31\textwidth}
			\begin{figure}
				\centering
				\includegraphics[ angle=0, scale=0.2]{images/velocidade_vertical}
				\label{ee}
				\caption{Velocidade vertical.}
			\end{figure}
		\end{minipage}
		\begin{minipage}[h!]{0.31\textwidth}
			\begin{figure}
				\centering
				\includegraphics[ angle=0, scale=0.2]{images/velocidade_horizontal}
				\label{eee}
				\caption{Velocidade horizontal.}
			\end{figure}
		\end{minipage}
		\begin{minipage}[h!]{0.31\textwidth}
			\begin{figure}
				\centering
				\includegraphics[ angle=0, scale=0.2]{images/campo_pressao}
				\label{eeeee}
				\caption{Pressão.}
			\end{figure}
		\end{minipage}
	
	O divergente ainda estava muito alto, revisou-se cuidadosamente o código e encontrou-se um erro na correção de pressão. Nesse momento, os valores de resíduo na continuidade diminuíram à ordem de grandeza do critério de parada da solução implícita.
	
	\end{frame}





	\begin{frame}
		\frametitle{Resultados Finais}
		\vspace{1cm}
		\begin{minipage}[h!]{0.31\textwidth}
			\begin{figure}
				\centering
				\includegraphics[ trim = {8.25cm 0cm 8.25cm 0cm}, clip , angle=0, scale=0.55]{images/resultados_finais_so_dinamicos}
				\label{ee}
				\caption{Velocidade vertical.}
			\end{figure}
		\end{minipage}
		\begin{minipage}[h!]{0.31\textwidth}
			\begin{figure}
				\centering
				\includegraphics[ trim = {0cm 0cm 16.7cm 0cm}, clip , angle=0, scale=0.55]{images/resultados_finais_so_dinamicos}
				\label{ee}
				\caption{Velocidade horizontal.}
			\end{figure}
		\end{minipage}
		\begin{minipage}[h!]{0.31\textwidth}
			\begin{figure}
				\centering
				\includegraphics[ trim = {16.7cm 0cm 0cm 0cm}, clip , angle=0, scale=0.55]{images/resultados_finais_so_dinamicos}
				\label{ee}
				\caption{Pressão.}
			\end{figure}
		\end{minipage}
		
	\end{frame}






	\begin{frame}
		\frametitle{Implementação térmica}
		Dessa forma, utilizou-se a formulação anteriormente desenvolvida para introduzir a convecção no código. Antes, a velocidade foi colocada como constante por todo o domínio, agora ela virá da simulação dinâmica. A partir da temperatura final se calculará a variação do $\rho$, que no próximo passo de tempo influenciará a solução dinâmica. Assim o acoplamento temperatura-velocidade-pressão ocorrerá da seguinte forma:
		
		\begin{figure}[h]
			\begin{center}
				\begin{tikzpicture}[node distance = 1.5cm, auto]
				\node[rect, rounded corners](step1) {$\vec{V}^n $};
				\node[rect,rounded corners , right of=step1, node distance=4.5cm] (step2) {$\vec{V}^{* n + 1} $};
				\node[rect, right of=step2, node distance=6.5cm] (step3) {$P^\prime$};
				\node[rect, rounded corners, below of=step3, node distance=2.5cm](step5){$ \vec{V}^{n+1}$};
				\node[rect, rounded corners, left of=step5, node distance=4cm](step6){$ \vec{T}^{n+1} $};
				\node[rect, rounded corners, left of=step6, node distance=4cm](step7){$ \rho^{n+1} $};
					
				\path[line](step1) -- node [above, text width=4em] {Resolução explicita.} node [below, text width=4em] {Eq.\ref{eqlabel1}} (step2);
					
				\path[line](step2) -- node [above, text width=4em] {Resolução implicita.} node (line2) [below, text width=4em] {Eq.\ref{eqlabel2}} (step3);
					
				\path[line](step3) -- node [right, near start,  text width=4em] {Explícito \\ /implícito} node [right, near end, text width=4em] {Eq.\ref{eqlabel3}} (step5);
				
				\path[line](step5) -- node [above, near start, text width=6em] {Explícito} node [below, near start, text width=6em] {/Implícito} (step6);
				
				\path[line](step6) -- node [above, near start, text width=6em] {Explícito} node [below, near start, text width=12em] {} (step7);
					
				\path[line](step7) -| node [right, near end, text width=6em] {Atualiza\\domínio\\dinâmico.} node [below, near start, text width=12em] {Checa continuidade.} (step1);
				\end{tikzpicture}
			\end{center}
			\caption{Fluxograma do loop principal do algoritmo.}
		\end{figure}
		
	\end{frame}



%%%%%%%%%%%%%%%%%%%%%%%%%%%%%%%%%%%%%%%%%%%%%%%%%%%%%%%%%%%%%%%%%%%%%%%%%%%%%%%%%%%%%%%%%%%%%%
\section{Desenvolvimento do programa}

	\begin{frame}
		\frametitle{Ideia geral do Software}
		
		\begin{figure}
			\centering
			\includegraphics[ trim = {0cm 0cm 0cm 0cm}, clip , angle=0, scale=0.38]{../src/main_4/Plano_de_projeto}
		\end{figure}

	\end{frame}


\begin{frame}
	\frametitle{Arquitetura da interface}
	
	\begin{figure}
		\centering
		\includegraphics[ trim = {0cm 0cm 0cm 0cm}, clip , angle=0, scale=0.38]{../src/main_4/Interface}
	\end{figure}
	
\end{frame}



\begin{frame}
	\frametitle{Ciclo de renderização}
	Foi necessário se desenvolver um loop de renderização, que é uma rotina chamada uma vez por frame, que tem como objetivos a aquisição de dados via protocolos MPI do processo responsável por gerir as simulações, renderização de toda ferramenta gráfica exibida na tela, assimilação de input do usuário e enviar estes dados, se necessário, a outros processos.
	
	\begin{figure}[h]
		\begin{center}
			\begin{tikzpicture}[node distance = 1.5cm, auto]
			\node[rect, rounded corners](step1) {$MPI DATA aquisition $};
			\node[rect,rounded corners , right of=step1, node distance=9cm] (step2) {$Rendering scene$};
			\node[rect, below of=step2, node distance=2.5cm] (step3) {$Acquires user input$};
			\node[rect, rounded corners, left of=step3, node distance=9cm](step5){$MPI Send user input$};
			
			
			\path[line](step1) -- node [above, text width=4em] {} node [below, text width=4em] {} (step2);
			
			\path[line](step2) -- node [above, text width=4em] {} node (line2) [below, text width=4em] {} (step3);
			
			\path[line](step3) -- node [right, near start,  text width=4em] {} node [right, near end, text width=4em] {} (step5);
			
			\path[line](step5) -- node [above, near start, text width=6em] {} node [below, near start, text width=6em] {} (step1);

			\end{tikzpicture}
		\end{center}
		\caption{Fluxograma do loop principal do algoritmo.}
	\end{figure}
	
\end{frame}






%%%%%%%%%%%%%%%%%%%%%%%%%%%%%%%%%%%%%%%%%%%%%%%%%%%%%%%%%%%%%%%%%%%%%%%%%%%%%%%%%%%%%%%%%%%%%%
\section{Agradecimentos}

	\begin{frame}
		\begin{center}
		\end{center}
		\placelogomflab
		\frametitle{Agradecimentos}
		\begin{center}
		\begin{tabular}{c c}
			{
				\includegraphics[trim=0.0cm 0.0cm 0.0cm 0.0cm,clip=true,height=0.2\textheight]{figuras_presentation_template/petrobras.png}
			}&{
				\includegraphics[trim=0.0cm 0.0cm 0.0cm 0.0cm,clip=true,height=0.2\textheight]{figuras_presentation_template/logo_mflab.png}}\\
				{\includegraphics[trim=0.0cm 0.0cm 0.0cm 0.0cm,clip=true,height=0.2\textheight]{figuras_presentation_template/cnpq.png}
			}&{
				\includegraphics[trim=0.0cm 0.0cm 0.0cm 0.0cm,clip=true,height=0.2\textheight]{figuras_presentation_template/CAPES.png}}\\
				{\includegraphics[trim=0.0cm 0.0cm 0.0cm 0.0cm,clip=true,height=0.2\textheight]{figuras_presentation_template/FAPEMIG.jpg}
			}&{
				\includegraphics[trim=0.0cm 0.0cm 0.0cm 0.0cm,clip=true,height=0.2\textheight]{figuras_presentation_template/UFU_black.jpg}
			}
		\end{tabular}
		\end{center}

	\end{frame}





\end{document}








%%%%%%%%%%%%%%%%%%%%%%%%%%%%%%%%%%%%%%% Exemplo de formatação de imagens
%		\begin{frame}
%			\frametitle{Adição de fronteiras extras}
%			\begin{tabular}{c c}
%
%				{\includegraphics[trim=0.0cm 0.0cm 0.0cm 0.0cm,clip=true,loop,height=0.5\textheight]{figuras/filtration_depois.png}}&{\includegraphics[trim=0.0cm 0.0cm 0.0cm 0.0cm,clip=true,loop,height=0.4\textheight]{figuras/filtration_depois_zoom.png}}\\
%
%			\end{tabular}
%
%		\end{frame}




%%%%%%%%%%%%%%%%%%%%%%%%%%%%%%%%%%%%%% Exemplo de formatação de imagens
%		\begin{frame}
%			\frametitle{Agora}
%			\centering
%			\begin{tabular}{c}
%
%				{\includegraphics[trim=0.00cm 2.0cm 0.0cm 2.0cm,clip=true,loop,width=0.9\textwidth]{figuras/t_x_51f.png}}\\{\includegraphics[trim=0.01cm 0.0cm 0.01cm 0.0cm,clip=true,loop,width=0.9\textwidth]{figuras/t_x_51999.png}}\\{\includegraphics[trim=0.01cm 0.0cm 0.01cm 0.0cm,clip=true,loop,width=0.9\textwidth]{figuras/t_x_51999g.png}}\\{\includegraphics[trim=0.01cm 0.0cm 0.01cm 0.0cm,clip=true,loop,width=0.9\textwidth]{figuras/t_x_51999y.png}}\\{\includegraphics[trim=0.01cm 0.0cm 0.01cm 0.0cm,clip=true,loop,width=0.9\textwidth]{figuras/t_x_51999b.png}}
%
%			\end{tabular}
%
%		\end{frame}





%%%%%%%%%%%%%%%%%%%%%%%%%%%%%%%%%%%%%  Formatação de equações:
%		\begin{frame}
%			\frametitle{Newton-Raphson}
%
%			\flushleft
%			Método de interface com jacobiano composto:
%
%			\centering
%			\begin{equation}\label{forte_eqNewton}
%			K(D+\Delta D) \approx K(D)+\Delta D \, J(D)
%			\end{equation}
%			\begin{equation}\label{forte_eqNewton2}
%			K(D) =  Estrutura(Fluido(D))-D =  0
%			\end{equation}
%			\begin{equation}\label{forte_eqNewton3}
%			J(D) =  Estrutura'(Fluido(D)) \, Fluido'(D)-I
%			\end{equation}
%			\begin{equation}\label{forte_eqNewton4}
%			Fluido(D): \mathbb{R}^{n} \to \mathbb{R}^{m}
%			\end{equation}
%
%			\flushleft
%			$Fluido'(D)$ é de tamanho $m x n$
%
%			\centering
%
%			\begin{equation}\label{forte_eqNewton5}
%			Estrutura(F): \mathbb{R}^{m} \to \mathbb{R}^{n}
%			\end{equation}
%
%			\flushleft
%			$Estrutura'(F)$ é de tamanho $n x m$\\
%			$Estrutura'(Fluido(D)) \, Fluido'(D)$ e $I$ é de tamanho $n x n$
%		\end{frame}




%%%%%%%%%%%%%%%%%%%%%%%%%%%%%%%%%%%%%%%%%%% Vários exemplos de formatação textual:


%		\begin{frame}
%			\frametitle{Conveniência do método de Multi Direct Forcing}
%
%			\flushleft
%			\textbf{Fraco:}\\
%			$\bullet$ Predição da velocidade.\\
%			$\bullet$ MDF. (Imposição da condição de dirichlet na interface e cálculo da força)\\
%			$\bullet$ Estrutura.\\
%			$\bullet$ Poisson.\\
%			$\bullet$ Correção de velocidade e pressão.\\ \\
%
%			\textbf{Forte:}\\
%			$\bullet$ Predição da velocidade.\\
%			while \\
%			\quad	$\longrightarrow$ MDF.\\
%			\quad	$\longrightarrow$ Estrutura.\\
%			end\\
%			$\bullet$ Poisson.\\
%			$\bullet$ Correção de velocidade e pressão.\\
%
%		\end{frame}



%
%%%%%%%%%%%%%%%%%%%%%%%%%%%%%%%%%  Modelo duas fotos lado a lado:


%		\begin{frame}
%		\frametitle{Limite do fraco}
%			ct=121
%			mi=200
%			\begin{tabular}{c c}
%			{\includegraphics[width=0.45\linewidth]{../../simulacoes_Estudo_dirigido2/fraco_mi_200_0_15_ct141/figuras/estrutura/vel_151}}&
%		   {\includegraphics[width=0.45\linewidth]{../../simulacoes_Estudo_dirigido2/fraco_mi_200_0_15_ct141/figuras/estrutura/vel_251}}\\
%		   {(a) Velocidade em linha centro da estrutura} & {(b) Velocidade transversal centro da estrutura}
%		\end{tabular}
%		\end{frame}



%%%%%%%%%%%%%%%%%%%%%%%%%%%%%%%%%%  Modelo tabela :

%		\begin{frame}
%			\frametitle{Comparação número de iterações}
%			\begin{tabular}{c c c c}
%				\hline
%				Método & Mínimo     &    Máximo &  Média\\ \hline
%				FPI MDF variável & 8     &    101 &  8.9764764764764760\\
%				FPI MDF fixo & 8     &     11 &  8.9099099099099099\\
%				QN Primeiro método de Broyden MDF variável & 18    &     101 &  18.281281281281281 \\ \hline
%			\end{tabular}
%		\end{frame}




