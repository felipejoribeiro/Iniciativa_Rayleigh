\documentclass[xcolor=dvipsnames,10pt,aspectratio=169]{beamer}
%\documentclass[xcolor=dvipsnames,10pt]{beamer}
\usepackage{etex}
\usepackage{pgf,pgfarrows,pgfnodes,pgfautomata,pgfheaps,pgfshade}
\usepackage[absolute,overlay]{textpos} 
%\usepackage{algorithm}
\usepackage{amsmath,amssymb}
\usepackage[utf8]{inputenc} 
\usepackage{colortbl}
\usepackage{graphicx} 
\usepackage[brazil]{babel}
\usepackage{tabularx} 
\usepackage{multirow}
\usepackage{booktabs}
\usepackage{listings}
%\usepackage{multimedia}
\usepackage{animate}
\usepackage{xcolor}
\usepackage{array}
\usepackage{longtable}
\usepackage{makecell}
\usepackage{caption}
\usetheme{Madrid} 
\usepackage{amsmath}
\usepackage{movie15}


\lstset{ %
%	backgroundcolor=\color{white},   % choose the background color; you must add \usepackage{color} or \usepackage{xcolor}
%	basicstyle=\footnotesize,        % the size of the fonts that are used for the code
	basicstyle=\scriptsize,        % the size of the fonts that are used for the code
	breakatwhitespace=false,         % sets if automatic breaks should only happen at whitespace
	breaklines=true,                 % sets automatic line breaking
	captionpos=t,                    % sets the caption-position to bottom
	commentstyle=\color{mygreen},    % comment style
	deletekeywords={...},            % if you want to delete keywords from the given language
	escapeinside={\%*}{*)},          % if you want to add LaTeX within your code
	extendedchars=true,              % lets you use non-ASCII characters; for 8-bits encodings only, does not work with UTF-8
%	frame=single,                    % adds a frame around the code
	keepspaces=true,                 % keeps spaces in text, useful for keeping indentation of code (possibly needs columns=flexible)
	keywordstyle=\color{blue},       % keyword style
%	language=make,                 % the language of the code
	morekeywords={*,...},            % if you want to add more keywords to the set
%	numbers=left,                    % where to put the line-numbers; possible values are (none, left, right)
%	numbersep=5pt,                   % how far the line-numbers are from the code
	numberstyle=\tiny\color{mygray}, % the style that is used for the line-numbers
	rulecolor=\color{black},         % if not set, the frame-color may be changed on line-breaks within not-black text (e.g. comments (green here))
	showspaces=false,                % show spaces everywhere adding particular underscores; it overrides 'showstringspaces'
	showstringspaces=false,          % underline spaces within strings only
	showtabs=false,                  % show tabs within strings adding particular underscores
	stepnumber=2,                    % the step between two line-numbers. If it's 1, each line will be numbered
}

\definecolor{mygreen}{rgb}{0,0.6,0}
\definecolor{mygray}{rgb}{0.5,0.5,0.5}
\definecolor{mymauve}{rgb}{0.58,0,0.82}

\usecolortheme{beaver}
\newcommand{\ul}{\underline}
\setbeamertemplate{footline}{\scriptsize{\vspace*{0.3cm}\hspace*{15cm}\insertframenumber\,/\,\inserttotalframenumber}}
\setbeamertemplate{caption}[numbered]
\setbeamerfont{caption}{size=\fontsize{8}{5}}

\setbeamercolor{block title}{	bg=Sepia , fg = White}
\setbeamercolor{block body}{bg=Brown!15, fg=Sepia }
\setbeamercolor{item projected}{bg=Sepia, fg=White}
\setbeamercolor{number projected}{bg = Black}

%declara as imagens usadas no layout do slide
\pgfdeclareimage[height=0.8cm]{mflab}{figuras_presentation_template/logo_mflab_transparente.png}
\pgfdeclareimage[height=1.0cm]{logoufu}{figuras_presentation_template/logo_ufu.jpg}
\pgfdeclareimage[height=1.0cm]{petro}{figuras_presentation_template/petrobras_2.png}

%posiciona o logotipo do MFLab
\setlength{\TPHorizModule}{1mm}
\setlength{\TPVertModule}{1mm}
\newcommand{\placelogomflab} 
{ 
	\begin{textblock}{13}(150.0,0.0)
		\pgfuseimage{mflab} 
	\end{textblock} 
	
% 	\begin{textblock}{13}(128.0,1.0)
% 		\pgfuseimage{logoufu} 
% 	\end{textblock} 
	
	\begin{textblock}{13}(150.0,70.0)
		\pgfuseimage{petro} 
	\end{textblock} 
}
%posiciona o logotipo do MFLab
\setlength{\TPHorizModule}{1mm}
\setlength{\TPVertModule}{1mm}
\newcommand{\placelogo} 
{ 
	\begin{textblock}{13}(150.0,0.0)
		\pgfuseimage{mflab} 
	\end{textblock} 
	
% 	\begin{textblock}{13}(128.0,1.0)
% 		\pgfuseimage{logoufu} 
% 	\end{textblock} 
	
	\begin{textblock}{13}(0.0,80.0)
		\pgfuseimage{petro} 
	\end{textblock} 
}

% \setlength{\TPHorizModule}{1mm}
% \setlength{\TPVertModule}{1mm}
% \newcommand{\placelogomflab_titulo} 
% { 
% 	\begin{textblock}{13}(150.0,0.0)
% 		\pgfuseimage{mflab} 
% 	\end{textblock} 
% 	
% 	\begin{textblock}{13}(0.0,0.0)
% 		\pgfuseimage{lmest} 
% 	\end{textblock} 
% 	
% % 	\begin{textblock}{13}(128.0,1.0)
% % 		\pgfuseimage{logoufu} 
% % 	\end{textblock} 
% 	
% 	\begin{textblock}{13}(75.0,80.0)
% 		\pgfuseimage{petro} 
% 	\end{textblock} 
% }



%insere o logotipo da ufu em todos os slides
% \logo{\includegraphics[height=0.8cm]{figuras/layout_slide/petrobras.png}}

\title{Iniciativa Rayleigh: Análise de escoamentos em cavidade, com simulação bidimensional e testes de modelos de turbulência com parametrizações inovadoras}

\author{ Felipe J. O. Ribeiro \\ \and \\ Orientador: Prof. Dr. Aristeu da Silveira Neto}

%\date{\tiny{02 de dezembro de 2015}}
\date{\tiny{\today}}
% \newcolumntype{M}[1]{>{\raggedright\arraybackslash}b{#1}}
% \newcolumntype{N}{@{}m{0pt}@{}}	
% \newcolumntype{M}{>{\begin{minipage}[b]{3cm}\raggedright{}}c<{\end{minipage}\minrowheight}}
% \setlength\extrarowheight{5pt}
\newcolumntype{C}[1]{>{\centering\let\newline\\\arraybackslash\hspace{0pt}}m{#1}}


\begin{document}

	\begin{frame}\placelogomflab
		\frametitle 
		{ \vfill
			\centering
			{
			\small{Universidade Federal de Uberlândia}\\
%			\small{Programa de Pós-Graduação em Engenharia Mecânica}\\
			\small{Laboratório de Mecânica dos Fluidos}\\
			}
		}
		\maketitle
	\end{frame}




	
\section<presentation>*{Sumário}
	
	\begin{frame}
		\frametitle{Sumário}\placelogomflab 
		{\scriptsize \tableofcontents}
	\end{frame}





	\AtBeginSection[]
	{
	\begin{frame}<beamer>
		\frametitle{Sumário}\placelogomflab 
		{\scriptsize \tableofcontents[current,currentsection]}
	\end{frame}
	}
	\AtBeginSubsection[]
	{
	\begin{frame}<beamer>
		  \frametitle{Sumário}\placelogomflab 
		  {\scriptsize \tableofcontents[current,currentsubsection]}
	\end{frame}
	}





%%%%%%%%%%%%%%%%%%%%%%%%%%%%%%%%%%%%%%%%%%%%%%%%%%%%%%%%%%%%%%%%%%%%%%%%%%%%%%%%%%%%%%%%%%%%%%
\section{Introdução}
	
	\begin{frame}
		\frametitle{Objetivos}
					
		\centering
		O presente texto procura documentar o desenvolvimento deste trabalho de pesquisa com o máximo de detalhes nas metodologias possível. Dês dos desenvolvimentos matemáticos e teóricos aos andamentos nos aspectos técnicos e organizacionais.
		
		\vspace{0.5cm}
		
		\flushleft
		Os tópicos centrais deste estudo são:\\
		\quad $\bullet$ Desenvolvimento teórico e matemático do escoamento bidimensional em cavidade.\\
		\quad $\bullet$ Implementação de modelo de turbulência baseado na hipótese de Boussinesq. \\
		\quad $\bullet$ Desenvolvimento das conversões dos parâmetros do Poiseuille para a cavidade.\\
		\quad $\bullet$ Aperfeiçoamento dos protocolos de visualização de dados com OpenGL.\\
		\quad $\bullet$ Desenvolvimento das rotinas em MPI no domínio bidimensional.\\
		
	\end{frame}





	\begin{frame}
		\frametitle{Estrutura de arquivos}
		\begin{tabular}{c c}
			{
				\begin{minipage}[h!]{0.15\textwidth}
					\centering
					\small
					Neste projeto, dês do início, será dada uma atenção especial à organização dos documentos, de forma a simplificar o andamento das atividades de documentação e prosseguimento.
					\vspace{6cm}
				\end{minipage}
			}&{
				\includegraphics[trim={0.0cm 0.0cm 0.0cm 0.0cm},clip=true, scale = 0.435]{../../arquivos.pdf}
			}
		\end{tabular}

	\end{frame}





	\begin{frame}
		\frametitle{Documento "Registro.tex"}
		\begin{tabular}{c c}
			{
				\begin{minipage}[h!]{0.4\textwidth}
					\centering
					\small
					Por questões de adequações á formatação em textos científicos, além desta apresentação também serão registrados os andamentos nesse documento em inglês. Ele segue o template da ASME para textos científicos. Assim, pretende-se já desenvolver as rotinas de visualização dos resultados de acordo com estas convenções. De forma a facilitar a criação de documentos para submissão em artigos e tornando este processo mais ágil.
					\vspace{6cm}
				\end{minipage}
			}&{
				\includegraphics[page = 1,trim={0.0cm 12.0cm 0.0cm 0.0cm},clip=true, scale = 0.4]{registro.pdf}
			}	
		\end{tabular}
	
	\end{frame}





	\begin{frame}
		\frametitle{Levante bibliográfico}
		\begin{minipage}[h!]{0.33\textwidth}
			\begin{figure}
				\centering
				\includegraphics[page = 1,trim={0.0cm 0.0cm 0.0cm 0.0cm},clip=true, scale = 0.15]{Referencias/An_introduction_to_Computational}
				\caption{Possui as referencias aos métodos matemáticos e numéricos de modelagem e discretização da fluidodinâmica ao domínio bidimensional.}	
			\end{figure}
			\vspace{6cm}
		\end{minipage}
		\begin{minipage}[h!]{0.33\textwidth}
			\begin{figure}
				\centering
				\includegraphics[page = 1,trim={0.0cm 0.0cm 0.0cm 0.0cm},clip=true, scale = 0.165]{Referencias/MPI_Guide_FORTRAN.png}\\	
				\caption{Um guia para se aprender a desenvolver em FORTRAN as rotinas em paralelo, com passagem de mensagem entre os processos.}
			\end{figure}
			\vspace{6cm}
		\end{minipage}
		\begin{minipage}[h!]{0.325\textwidth}
			\begin{figure}
				\centering
				\includegraphics[page = 1,trim={1.0cm 1.0cm 1.0cm 0.0cm},clip=true, scale = 0.165]{Referencias/Projeto_cebeci}\\
				\caption{Artigo previamente publicado, para resgate dos ajustes paramétricos.}
			\end{figure}
			\vspace{6cm}
		\end{minipage}
	
	\end{frame}
	
	
	
	

%%%%%%%%%%%%%%%%%%%%%%%%%%%%%%%%%%%%%%%%%%%%%%%%%%%%%%%%%%%%%%%%%%%%%%%%%%%%%%%%%%%%%%%%%%%%%%
\section{Modelo físico}

	\begin{frame}
		\frametitle{Hipóteses e considerações sobre o sistema}
		\begin{tabular}{c c}
			{
				\begin{minipage}{0.4\textwidth}
					\small
					\centering
					O problema fluidodinâmico escolhido foi o da cavidade. Ele consiste em uma quantidade de fluido confinada a um sistema termodinâmico com duas fontes de energia térmica como observado na Fig.\ref{sistema_termico_1}.
					
					\vspace{0.2cm}
					
					\flushleft
					As considerações feitas foram:\\
					\quad $\bullet$ O fluido é incompressível e newtoniano, sendo que sua massa específica varia somente em função da temperatura.\\
					\quad $\bullet$ O sistema foi considerado auto-similar na coordenada $z$. \\
					\quad $\bullet$ As superfícies perpendiculares ao eixo $y$ são isoladas termicamente.\\
					\quad $\bullet$ Considera-se uma fonte fria e uma fonte quente nas superfícies perpendiculares ao eixo $x$.\\
				\end{minipage}
			}&{
				\begin{minipage}{0.5\textwidth}
					\begin{figure}
						\centering
						\includegraphics[page = 1,trim={6.0cm 2.0cm 6.0cm 1.5cm},clip=true, scale = 0.4]{images/Cavidade_fisico_1.pdf}
						\caption{Sistema físico da cavidade.}
						\label{sistema_termico_1}
					\end{figure}
				\end{minipage}
			}
		\end{tabular}
	
	\end{frame}
	
	
	
	
	
	\begin{frame}
		\frametitle{Sobre Reynolds e Rayleigh}
		\centering
		Como os ajustes foram feitos dentro de um range de números de Reynolds, é necessário que se faça um estudo de como achar estes valores em número de Rayleigh, ou seja, com aplicação em convecção natural.
		
	\end{frame}





%%%%%%%%%%%%%%%%%%%%%%%%%%%%%%%%%%%%%%%%%%%%%%%%%%%%%%%%%%%%%%%%%%%%%%%%%%%%%%%%%%%%%%%%%%%%%%
\section{Desenvolvimento matemático}

	\begin{frame}
		\frametitle{Equações representativas}
		\flushleft
		Para este desenvolvimento matemático serão utilizadas:\\
		\quad $\bullet$ As equações de Navier Stokes.\\
		\quad $\bullet$ A equação da continuidade.\\
		\quad $\bullet$ A equação de balanço da energia térmica.\\
		\quad $\bullet$ Uma equação de estado.\\
		
		\begin{equation}
		\nabla . \vec{V} = 0
		\end{equation}
		\begin{equation}
		\frac{\partial T}{\partial t} + \vec{\nabla} . \left( \vec{V} \vec{V} \right) =  -\frac{1}{\rho_o}\vec{\nabla}P + \frac{\rho - \rho_o}{\rho_o} \vec{g} + \nu \nabla ^2 \vec{V}
		\end{equation}
		\begin{equation}
		\frac{\partial\vec{V}}{\partial t} + \vec{\nabla} . \left( \vec{V}T \right) = \alpha \nabla^2T
		\end{equation}
		\begin{equation}
		\frac{\rho - \rho_o}{\rho_o} = \beta \left( T - T_o\right)
		\end{equation}
		
		
		
		
		
		
	\end{frame}





%%%%%%%%%%%%%%%%%%%%%%%%%%%%%%%%%%%%%%%%%%%%%%%%%%%%%%%%%%%%%%%%%%%%%%%%%%%%%%%%%%%%%%%%%%%%%%	
\section{Agradecimentos}
			
	\begin{frame}	
		\begin{center}
		Obrigado às entidades cujo auxílio fora inestimável ao andamento deste trabalho. 
		\end{center}
		\placelogomflab 
		\frametitle{Agradecimentos}
		\begin{center}
		\begin{tabular}{c c}
			{
				\includegraphics[trim=0.0cm 0.0cm 0.0cm 0.0cm,clip=true,height=0.2\textheight]{figuras_presentation_template/petrobras.png}
			}&{
				\includegraphics[trim=0.0cm 0.0cm 0.0cm 0.0cm,clip=true,height=0.2\textheight]{figuras_presentation_template/logo_mflab.png}}\\
				{\includegraphics[trim=0.0cm 0.0cm 0.0cm 0.0cm,clip=true,height=0.2\textheight]{figuras_presentation_template/cnpq.png}
			}&{
				\includegraphics[trim=0.0cm 0.0cm 0.0cm 0.0cm,clip=true,height=0.2\textheight]{figuras_presentation_template/CAPES.png}}\\
				{\includegraphics[trim=0.0cm 0.0cm 0.0cm 0.0cm,clip=true,height=0.2\textheight]{figuras_presentation_template/FAPEMIG.jpg}
			}&{
				\includegraphics[trim=0.0cm 0.0cm 0.0cm 0.0cm,clip=true,height=0.2\textheight]{figuras_presentation_template/UFU_black.jpg}
			}
		\end{tabular}
		\end{center}
			
	\end{frame}
		
			
		
		
		
\end{document}




		



%%%%%%%%%%%%%%%%%%%%%%%%%%%%%%%%%%%%%%% Exemplo de formatação de imagens		
%		\begin{frame}
%			\frametitle{Adição de fronteiras extras}
%			\begin{tabular}{c c}
%				
%				{\includegraphics[trim=0.0cm 0.0cm 0.0cm 0.0cm,clip=true,loop,height=0.5\textheight]{figuras/filtration_depois.png}}&{\includegraphics[trim=0.0cm 0.0cm 0.0cm 0.0cm,clip=true,loop,height=0.4\textheight]{figuras/filtration_depois_zoom.png}}\\
%				
%			\end{tabular}
%			
%		\end{frame}




%%%%%%%%%%%%%%%%%%%%%%%%%%%%%%%%%%%%%% Exemplo de formatação de imagens		
%		\begin{frame}
%			\frametitle{Agora}
%			\centering
%			\begin{tabular}{c}
%				
%				{\includegraphics[trim=0.00cm 2.0cm 0.0cm 2.0cm,clip=true,loop,width=0.9\textwidth]{figuras/t_x_51f.png}}\\{\includegraphics[trim=0.01cm 0.0cm 0.01cm 0.0cm,clip=true,loop,width=0.9\textwidth]{figuras/t_x_51999.png}}\\{\includegraphics[trim=0.01cm 0.0cm 0.01cm 0.0cm,clip=true,loop,width=0.9\textwidth]{figuras/t_x_51999g.png}}\\{\includegraphics[trim=0.01cm 0.0cm 0.01cm 0.0cm,clip=true,loop,width=0.9\textwidth]{figuras/t_x_51999y.png}}\\{\includegraphics[trim=0.01cm 0.0cm 0.01cm 0.0cm,clip=true,loop,width=0.9\textwidth]{figuras/t_x_51999b.png}}
%				
%			\end{tabular}
%			
%		\end{frame}





%%%%%%%%%%%%%%%%%%%%%%%%%%%%%%%%%%%%%  Formatação de equações:		
%		\begin{frame}
%			\frametitle{Newton-Raphson}
%			
%			\flushleft
%			Método de interface com jacobiano composto:
%			
%			\centering
%			\begin{equation}\label{forte_eqNewton}
%			K(D+\Delta D) \approx K(D)+\Delta D \, J(D)
%			\end{equation}
%			\begin{equation}\label{forte_eqNewton2}
%			K(D) =  Estrutura(Fluido(D))-D =  0
%			\end{equation}
%			\begin{equation}\label{forte_eqNewton3}
%			J(D) =  Estrutura'(Fluido(D)) \, Fluido'(D)-I
%			\end{equation}
%			\begin{equation}\label{forte_eqNewton4}
%			Fluido(D): \mathbb{R}^{n} \to \mathbb{R}^{m}
%			\end{equation}
%			
%			\flushleft
%			$Fluido'(D)$ é de tamanho $m x n$
%			
%			\centering
%			
%			\begin{equation}\label{forte_eqNewton5}
%			Estrutura(F): \mathbb{R}^{m} \to \mathbb{R}^{n}
%			\end{equation}
%			
%			\flushleft
%			$Estrutura'(F)$ é de tamanho $n x m$\\
%			$Estrutura'(Fluido(D)) \, Fluido'(D)$ e $I$ é de tamanho $n x n$
%		\end{frame}




%%%%%%%%%%%%%%%%%%%%%%%%%%%%%%%%%%%%%%%%%%% Vários exemplos de formatação textual:		


%		\begin{frame}
%			\frametitle{Conveniência do método de Multi Direct Forcing}
%			
%			\flushleft
%			\textbf{Fraco:}\\
%			$\bullet$ Predição da velocidade.\\
%			$\bullet$ MDF. (Imposição da condição de dirichlet na interface e cálculo da força)\\
%			$\bullet$ Estrutura.\\
%			$\bullet$ Poisson.\\
%			$\bullet$ Correção de velocidade e pressão.\\ \\
%
%			\textbf{Forte:}\\
%			$\bullet$ Predição da velocidade.\\
%			while \\
%			\quad	$\longrightarrow$ MDF.\\
%			\quad	$\longrightarrow$ Estrutura.\\
%			end\\
%			$\bullet$ Poisson.\\
%			$\bullet$ Correção de velocidade e pressão.\\
%
%		\end{frame}

		

%		
%%%%%%%%%%%%%%%%%%%%%%%%%%%%%%%%%  Modelo duas fotos lado a lado:


%		\begin{frame}
%		\frametitle{Limite do fraco}
%			ct=121
%			mi=200
%			\begin{tabular}{c c}
%			{\includegraphics[width=0.45\linewidth]{../../simulacoes_Estudo_dirigido2/fraco_mi_200_0_15_ct141/figuras/estrutura/vel_151}}&
%		   {\includegraphics[width=0.45\linewidth]{../../simulacoes_Estudo_dirigido2/fraco_mi_200_0_15_ct141/figuras/estrutura/vel_251}}\\
%		   {(a) Velocidade em linha centro da estrutura} & {(b) Velocidade transversal centro da estrutura}
%		\end{tabular}
%		\end{frame}



%%%%%%%%%%%%%%%%%%%%%%%%%%%%%%%%%%  Modelo tabela :

%		\begin{frame}
%			\frametitle{Comparação número de iterações}
%			\begin{tabular}{c c c c}
%				\hline
%				Método & Mínimo     &    Máximo &  Média\\ \hline
%				FPI MDF variável & 8     &    101 &  8.9764764764764760\\
%				FPI MDF fixo & 8     &     11 &  8.9099099099099099\\
%				QN Primeiro método de Broyden MDF variável & 18    &     101 &  18.281281281281281 \\ \hline
%			\end{tabular}
%		\end{frame}	




